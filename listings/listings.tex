\documentclass[article,a4paper,oneside,10pt]{memoir}

% -------------------------------------------------------------------------- %
% Page Layout                                                                %
% -------------------------------------------------------------------------- %
\isopage
\setlrmarginsandblock{0.142857111\paperwidth}{0.142857111\paperwidth}{*}
\setulmarginsandblock{0.111111111\paperheight}{*}{1.25}
\checkandfixthelayout
\newfixedcaption{\figcaption}{figure}
\newfixedcaption{\tabcaption}{table}
\captiontitlefont{\small}
\captionnamefont{\bfseries\small}
\captiondelim{: }

% -------------------------------------------------------------------------- %
% Bibliography                                                               %
% -------------------------------------------------------------------------- %
\renewcommand{\bibsection}{%
    \chapter{References}%
    \prebibhook}
\bibintoc
\renewcommand{\cftdot}{\textcolor{gray}{.}}
\renewcommand{\cftchapterdotsep}{\cftdotsep} % Chapters have dots in ToC
\renewcommand{\cftsectiondotsep}{\cftnodots} % Sections have no dots in ToC


% -------------------------------------------------------------------------- %
% Chapter and Section Headings                                               %
% -------------------------------------------------------------------------- %
\makeatletter
\makechapterstyle{alpensec}{%
  \chapterstyle{default}
  \setlength{\beforechapskip}{3.5ex \@plus 1ex \@minus .2ex}
  \renewcommand*{\chapterheadstart}{\vspace{\beforechapskip}}
  \setlength{\afterchapskip}{2.3ex \@plus .2ex}
  \renewcommand{\printchaptername}{}
  \renewcommand{\chapternamenum}{}
  \renewcommand{\chaptitlefont}{\sffamily\LARGE\bfseries}
  \renewcommand{\chapnumfont}{\chaptitlefont}
  \renewcommand{\printchapternum}{\chapnumfont \thechapter\quad}
  \renewcommand{\afterchapternum}{}}

\renewcommand{\section}{%
  \sechook%
  \@startsection{section}{1}%  level 1
      {\secindent}%            heading indent
      {\beforesecskip}%        skip before the heading
      {\aftersecskip}%         skip after the heading
      {\sffamily\secheadstyle}} % font

\makeatother

\chapterstyle{alpensec}


% -------------------------------------------------------------------------- %
% Packages                                                                   %
% -------------------------------------------------------------------------- %
\usepackage[%
        bookmarksnumbered=true,
        colorlinks=true,
        linkcolor=cyan!50!blue,
        citecolor=violet,
        urlcolor=purple,
        %hidelinks=false,
    ]{hyperref}
\usepackage[light,nott]{kpfonts}
\usepackage{microtype}
\usepackage[english]{babel}
\usepackage[T1]{fontenc}
\usepackage[utf8]{inputenc}
\usepackage{xcolor-solarized}
\usepackage{minted}
\usepackage{tcolorbox}
\tcbuselibrary{minted}
\usepackage{verbatimbox}


% -------------------------------------------------------------------------- %
% Macros                                                                     %
% -------------------------------------------------------------------------- %
\newcommand\code[1]{\texttt{#1}}


% -------------------------------------------------------------------------- %
% Title                                                                      %
% -------------------------------------------------------------------------- %
\title{\textsf{\Huge Verbatim Text and Code Listings in \LaTeX}}
\author{Raphael Frey, <\href{mailto:webmaster@alpenwasser.net}{\nolinkurl{webmaster@alpenwasser.net}}>}
\date{\vspace{1em}\today}


% ************************************************************************** %
\begin{document}
% ************************************************************************** %

\maketitle

\begin{abstract}
    There  are  many ways  to  integrate  verbatim  text (text  which  should,
    ideally,  be unaltered)  and  formatted code  (listings)  into a  \LaTeX{}
    document,  as  can  easily  be  seen when  looking  at  the  corresponding
    topic  pages on  CTAN \cite{ctan:topic:listings,ctan:topic:verbatim}. This
    document  presents  examples  for  a  few of  them. The  idea  is  not  to
    provide  in-depth  documentation,  but  to  give a  brief  overview  of  a
    few  possibilities.   Consulting  the  respective user  manual  is  always
    recommended for detailled information.

    For  the  \emph{very}  impatient: For  most   code  listings,  I  use  the
    \code{minted}  package these  days. While it  is slow  (at least  on first
    compilation),  it has  served  me well  most of  the  time. But there  are
    alternatives,  and personal  preferences  matter,  too. One's mileage  may
    vary.
\end{abstract}

\tableofcontents*


% ========================================================================== %
\newpage
\chapter{The \code{verbatim} Environment}
\label{chap:verbatim}
% ========================================================================== %

\LaTeX{} provides a \code{verbatim} Environment which, although it has quite a
few limitations, is simple to use and often gets the job done well enough. For
fancier things, there exists a re-implementation with a few added niceties.


% -------------------------------------------------------------------------- %
\section{Stock \LaTeX}
\label{sec:verbatim:stock}
% -------------------------------------------------------------------------- %

The stock \LaTeX{} \code{verbatim} environment is used like this:

\begin{tcblisting}{listing side text,title=\textsf{\bfseries LaTeX Result vs. Code},listing engine=minted,minted language=text}
\begin{verbatim}
#include <stdio.h>

main() {
    print("Hello, world!\n");
}
\end{verbatim}
\end{tcblisting}


% -------------------------------------------------------------------------- %
\section{Re-Implementation \cite{verbatim}}
\label{sec:verbatim:reimp}
% -------------------------------------------------------------------------- %

The re-implementation alleviates some limitations of the stock \code{verbatim}
environment. For  more  details  and  to evaluate  whether  these  limitations
may  be  of  relevance  to  your   use  case,  consult  the  documentation  at
\cite{verbatim}. Basic  usage  is  identical   to  the  stock  \code{verbatim}
environment.

\begin{tcblisting}{listing side text,title=\textsf{\bfseries LaTeX Result vs. Code},listing engine=minted,minted language=text}
\begin{verbatim}
#include <stdio.h>

main() {
    print("Hello, world!\n");
}
\end{verbatim}
\end{tcblisting}



% ========================================================================== %
\chapter{The \code{verbatimbox} Package \cite{verbatimbox}}
\label{chap:verbatimbox}
% ========================================================================== %

A  box  is   defined  via  \mintinline{text}{\begin{verbbox}[options]  content
\end{verbbox}}. This, however, does not yet  display the bxo. This can be done
via  the \mintinline{text}{\theverbbox}  command. This can  be easily  wrapped
inside a box like so:

\begin{tcblisting}{listing side text,title=\textsf{\bfseries LaTeX Result vs. Code},listing engine=minted,minted language=text}
\begin{verbbox}
#include <stdio.h>

main() {
    print("Hello, world!\n");
}
\end{verbbox}
\fbox{\theverbbox}
\end{tcblisting}

Note the removal of empty lines in the result on the right.As above, check the
documentation \cite{verbatimbox} for more options and possibilities.


% ========================================================================== %
\chapter{The \code{fancyvrb} Package \cite{fancyvrb}}
\label{chap:fancyvrb}
% ========================================================================== %


% ========================================================================== %
\chapter{The \code{listing} Package \cite{listing}}
\label{chap:listing}
% ========================================================================== %


% ========================================================================== %
\chapter{The \code{listings}      Package \cite{listings}}
\label{chap:listings}
% ========================================================================== %


% ========================================================================== %
\chapter{The \code{listingsutf8}  Package \cite{listingsutf8}}
\label{chap:listingsutf8}
% ========================================================================== %


% ========================================================================== %
\chapter{The \code{mlpretty}      Package \cite{mlpretty}}
\label{chap:mlpretty}
% ========================================================================== %


% ========================================================================== %
\chapter{The \code{minted}        Package \cite{minted}}
\label{chap:minted}
% ========================================================================== %



% ========================================================================== %
\vfill
\begin{thebibliography}{1}
% ========================================================================== %

    \bibitem{verbatim}
        Rainer Sch\"opf and The \LaTeX{} Team.
        ``\emph{verbatim -– Reimplementation of and extensions to \LaTeX{} verbatim}'',
        Version 1.5q,
        2001-MAR-12.
        [Online],
        \href{http://ctan.org/pkg/verbatim}{\nolinkurl{http://ctan.org/pkg/verbatim}},
        [Accessed: 2017-MAR-22].

    \bibitem{verbatimbox}
        Steven B. Segletes.
        ``\emph{verbatimbox – Deposit verbatim text in a box}'',
        Version 3.13,
        2014-MAR-12.
        [Online],
        \href{http://ctan.org/pkg/verbatimbox}{\nolinkurl{http://ctan.org/pkg/verbatimbox}},
        [Accessed: 2017-MAR-22].

    \bibitem{fancyvrb}
        Timothy Van Zandt and Herbert Vo\ss{} and Denis Girou.
        ``\emph{fancyrvb -- Sophisticated verbatim text}'',
        Version 2.8,
        2010-MAY-15.
        [Online],
        \href{http://ctan.org/pkg/fancyvrb}{\nolinkurl{http://ctan.org/pkg/fancyvrb}},
        [Accessed: 2017-MAR-22].

    \bibitem{listing}
        Volker Kuhlmann and Matthew Hebley.
        ``\emph{listing -- Produce formatted program listings}'',
        Version 1.2,
        1999-MAY-25.
        [Online],
        \href{http://ctan.org/pkg/listing}{\nolinkurl{http://ctan.org/pkg/listing}},
        [Accessed: 2017-MAR-22].

    \bibitem{listings}
        Brooks Moses and Carsten Heinz and Jobst Hoffman.
        ``\emph{listings -- Typeset source code listings using \LaTeX}'',
        Version 1.6,
        2015-JUN-04.
        [Online],
        \href{http://ctan.org/pkg/listings}{\nolinkurl{http://ctan.org/pkg/listings}},
        [Accessed: 2017-MAR-22].

    \bibitem{listingsutf8}
        Heiko Oberdiek.
        ``\emph{listingsutf8 -- Allow UTF-8 in listings input}'',
        Version 1.3,
        2016-MAY-16.
        [Online],
        \href{http://ctan.org/pkg/listingsutf8}{\nolinkurl{http://ctan.org/pkg/listingsutf8}},
        [Accessed: 2017-MAR-22].

    \bibitem{mlpretty}
        Julien Cretel.
        ``\emph{matlab-prettifier -- Pretty-print Matlab source code}'',
        Version 0.3,
        2014-JUN-19.
        [Online],
        \href{http://ctan.org/pkg/matlab-prettifier}{\nolinkurl{http://ctan.org/pkg/matlab-prettifier}},
        [Accessed: 2017-MAR-22].

    \bibitem{minted}
        Konrad Rudolph and Geoffrey Poore.
        ``\emph{minted -- Highlighted source code for \LaTeX}'',
        Version 2.4.1,
        2016-OCT-31.
        [Online],
        \href{http://ctan.org/pkg/minted}{\nolinkurl{http://ctan.org/pkg/minted}},
        [Accessed: 2017-MAR-22].

    \bibitem{verbments}
        Dejan Živkovic.
        ``\emph{verbments -- Syntax highlighting of source code in \LaTeX{} documents}'',
        Version 1.2,
        2011-AUG-20.
        [Online],
        \href{http://ctan.org/pkg/verbments}{\nolinkurl{http://ctan.org/pkg/verbments}},
        [Accessed: 2017-MAR-22].

    \bibitem{ctan:topic:listings}
        Comprehensive \TeX{} Archive Network.
        ``\emph{Topic listing}''.
        [Online],
        \href{http://ctan.org/topic/listing}{\nolinkurl{http://ctan.org/topic/listing}},
        [Accessed: 2017-MAR-22].

    \bibitem{ctan:topic:verbatim}
        Comprehensive \TeX{} Archive Network.
        ``\emph{Topic verbatim}''.
        [Online],
        \href{http://ctan.org/topic/verbatim}{\nolinkurl{http://ctan.org/topic/verbatim}},
        [Accessed: 2017-MAR-22].


\end{thebibliography}

\end{document}
