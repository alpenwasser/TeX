\documentclass[article,a4paper,oneside,10pt]{memoir}

% -------------------------------------------------------------------------- %
% Page Layout                                                                %
% -------------------------------------------------------------------------- %
\setlrmarginsandblock{0.142857111\paperwidth}{0.142857111\paperwidth}{*}
\setulmarginsandblock{0.111111111\paperheight}{*}{1.25}
\checkandfixthelayout
\newfixedcaption{\figcaption}{figure}
\newfixedcaption{\tabcaption}{table}
\captiontitlefont{\small}
\captionnamefont{\bfseries\small}
\captiondelim{: }

% -------------------------------------------------------------------------- %
% Bibliography                                                               %
% -------------------------------------------------------------------------- %
\renewcommand{\bibsection}{%
    \chapter{References}%
    \prebibhook}
\bibintoc
\renewcommand{\cftdot}{\textcolor{gray}{.}}
\renewcommand{\cftchapterdotsep}{\cftdotsep} % Chapters have dots in ToC
\renewcommand{\cftsectiondotsep}{\cftnodots} % Sections have no dots in ToC


% -------------------------------------------------------------------------- %
% Chapter and Section Headings                                               %
% -------------------------------------------------------------------------- %
\makeatletter
\makechapterstyle{alpensec}{%
  \chapterstyle{default}
  \setlength{\beforechapskip}{3.5ex \@plus 1ex \@minus .2ex}
  \renewcommand*{\chapterheadstart}{\vspace{\beforechapskip}}
  \setlength{\afterchapskip}{2.3ex \@plus .2ex}
  \renewcommand{\printchaptername}{}
  \renewcommand{\chapternamenum}{}
  \renewcommand{\chaptitlefont}{\sffamily\LARGE\bfseries}
  \renewcommand{\chapnumfont}{\chaptitlefont}
  \renewcommand{\printchapternum}{\chapnumfont \thechapter\quad}
  \renewcommand{\afterchapternum}{}}

\renewcommand{\section}{%
  \sechook%
  \@startsection{section}{1}%  level 1
      {\secindent}%            heading indent
      {\beforesecskip}%        skip before the heading
      {\aftersecskip}%         skip after the heading
      {\sffamily\secheadstyle}} % font

\makeatother

\chapterstyle{alpensec}


% -------------------------------------------------------------------------- %
% Packages                                                                   %
% -------------------------------------------------------------------------- %
\usepackage[%
        bookmarksnumbered=true,
        colorlinks=true,
        linkcolor=cyan!50!blue,
        citecolor=violet,
        urlcolor=purple,
        %hidelinks=false,
    ]{hyperref}
\usepackage[light,nott]{kpfonts}
\usepackage{microtype}
\usepackage[english]{babel}
\usepackage[T1]{fontenc}
\usepackage[utf8]{inputenc}
\usepackage{xcolor-solarized}
\usepackage{minted}
\usepackage{packages/mylisting}
\usepackage{tcolorbox}
\tcbuselibrary{minted}
\usepackage{verbatimbox}
\usepackage{fancyvrb}
\DefineVerbatimEnvironment%
    {GlobalCustomVerbatim}{Verbatim}{%
    gobble=8,
    numbers=left,
    numbersep=4pt,
    frame=single,
    framerule=1pt,
    framesep=1ex,
    rulecolor=\color{cyan!50!blue}}
\usepackage{listings}


% -------------------------------------------------------------------------- %
% Macros                                                                     %
% -------------------------------------------------------------------------- %
\newcommand\code[1]{\texttt{#1}}


% -------------------------------------------------------------------------- %
% Title                                                                      %
% -------------------------------------------------------------------------- %
\title{\textsf{\Huge Verbatim Text and Code Listings in \LaTeX}}
\author{Raphael Frey, <\href{mailto:webmaster@alpenwasser.net}{\nolinkurl{webmaster@alpenwasser.net}}>}
\date{\vspace{1em}\today}

\tcbset{
    listing and text,
    listing engine=minted,
    title=\textsf{\bfseries Code \hfill \LaTeX{} Output},
    minted language=text,
    minted options={gobble=4}}

% ************************************************************************** %
\begin{document}
% ************************************************************************** %


\maketitle

\begin{abstract}
    There  are   many  ways  to   integrate  verbatim  text  (text   which  is
    presented  in  the  resulting  document  as it  is  in  the  source  code)
    and  listings  (which may  include  pretty  printing, syntax  highlighting
    and  other  fancy  things)  into   a  \LaTeX{}  document,  as  can  easily
    be  seen   when  looking  at   the  corresponding  topic  pages   on  CTAN
    \cite{ctan:topic:listings,ctan:topic:verbatim}. This   document   presents
    examples  for  a  few  of  them. The  idea  is  not  to  provide  in-depth
    documentation, but  to give  a brief  overview of  a few  possibilities to
    enable the  reader to  get an  idea of  what's out  there and  what might,
    potentially, be useful to them.   Consulting the respective user manual is
    always recommended for  more detailled information and  achieving the best
    results.

    For  the  \emph{very}  impatient: For  most   code  listings,  I  use  the
    \code{minted}  package these  days. While it  is slow  (at least  on first
    compilation),  it has  served  me well  most of  the  time. But there  are
    alternatives  (hence  this  document), and  personal  preferences  matter,
    too. One's mileage may vary.
\end{abstract}

\tableofcontents*


% ========================================================================== %
\newpage
\chapter{The \code{verbatim} Environment and \code{\textbackslash{}verb} Command}
\label{chap:verbatim}
% ========================================================================== %

\LaTeX{} provides a \code{verbatim} Environment which, although it has quite a
few  limitations,  is  simple  to  use  and  often  gets  the  job  done  well
enough. For fancier things, there exists  a re-implementation with a few added
niceties. For inline code, there is the \code{\textbackslash{}verb} command.


% -------------------------------------------------------------------------- %
\section{Stock \LaTeX}
\label{sec:verbatim:stock}
% -------------------------------------------------------------------------- %

The stock \LaTeX{} \code{verbatim} environment is used like this:

\begin{tcblisting}{listing side text}
    \begin{verbatim}
    #include <stdio.h>

    main() {
        print("Hello, world!\n");
    }
    \end{verbatim}
\end{tcblisting}

Writing inline code can be done with the \code{\textbackslash{}verb} command:

\begin{tcblisting}{listing side text}
    Inline  verbatim content  is  
    \verb|typeset inside regular text| 
    instead of being put in a special box or
    paragraph or being similarly separated.
\end{tcblisting}

% -------------------------------------------------------------------------- %
\section{Re-Implementation of the \code{verbatim} and \code{verbatim*} Environments \cite{verbatim}}
\label{sec:verbatim:reimp}
% -------------------------------------------------------------------------- %

The re-implementation alleviates some limitations of the stock \code{verbatim}
environment. For  more  details  and  to evaluate  whether  these  limitations
may  be  of  relevance  to  your   use  case,  consult  the  documentation  at
\cite{verbatim}. Basic  usage  is  identical   to  the  stock  \code{verbatim}
environment.

\begin{tcblisting}{listing side text}
    \begin{verbatim}
    #include <stdio.h>

    main() {
        print("Hello, world!\n");
    }
    \end{verbatim}
\end{tcblisting}


% ========================================================================== %
\newpage
\chapter{The \code{verbatimbox} Package \cite{verbatimbox}}
\label{chap:verbatimbox}
% ========================================================================== %

A  box  is   defined  via  \mintinline{text}{\begin{verbbox}[options]  content
\end{verbbox}}.  This, however,  does not yet display the  box. Instead, it is
stored for later use  via the \mintinline{text}{\theverbbox} command. This can
be easily  wrapped inside an \mintinline{text}{\fbox},  optionally adding some
vertical space with the \mintinline{text}{\addvbuffer} command.

\vfill

\begin{tcblisting}{}
    \begin{verbbox}[\arabic{VerbboxLineNo}:\hspace{1ex}]
    #include <stdio.h>

    main() {
        print("Hello, world!\n");
    }
    \end{verbbox}
    \fbox{\addvbuffer[10pt 5pt]\theverbbox} % different spacing above and below
    \fbox{\addvbuffer[10pt]\theverbbox}     % same spacing above and below
\end{tcblisting}

\vfill

Note the removal of empty lines in the result on the right.As above, check the
documentation \cite{verbatimbox} for more options and possibilities. Also, the
optional argument  \mintinline{text}{[\arabic{VerbboxLineNo}:\hspace{1ex}]} to
\mintinline{text}{\begin{verbbox}} must be  on the same line and  only on that
line.

For cases where the limitation of having  only a single named box to which one
can refer, \mintinline{text}{\theverbbox}, must  be overcome, there exists the
\mintinline{text}{\myverbbox} command, by which named boxes can be created for
later  use. One use  case for  this  is the  use of  verbatim environments  in
tabular environments, where the use of \code{verbatim} is not allowed (example
from the \code{verbatimbox} manual \cite{verbatimbox}).

\vfill

\begin{tcblisting}{}
    \begin{myverbbox}{\vtheta}\theta\end{myverbbox}
    \begin{myverbbox}{\valpha}\alpha\end{myverbbox}
    \begin{tabular}{|c|c|} \hline
    \valpha & $\alpha$ \\ \hline
    \vtheta & $\theta$ \\ \hline
    \end{tabular}
\end{tcblisting}

\newpage
\code{verbatimbox} also allows including the contents of an entire file:

\begin{tcblisting}{listing side text}
    \verbfilebox{code/hw.c}
    \theverbbox
\end{tcblisting}

Because  \code{verbatimbox}  uses boxes  to  save  its contents,  they  cannot
be  broken across  pages. For  breaking verbatims  across  pages, there  esist
the  \code{verbnobox}  environment and  the  \mintinline{text}{\verbfilenobox}
command. These do not  allow recalling the verbatim contents at  a later point
though; their output is generated directly where they are placed in the source
code.


\begin{tcblisting}{}
    \begin{verbnobox}[\arabic{VerbboxLineNo}:\hspace{1ex}]
    #include <stdio.h>

    main() {
        print("Hello, world!\n");
    }
    \end{verbnobox}
\end{tcblisting}

\begin{tcblisting}{listing side text}
    \verbfilenobox{code/hw.c}
\end{tcblisting}

% ========================================================================== %
\newpage
\chapter{The \code{fancyvrb} Package \cite{fancyvrb}}
\label{chap:fancyvrb}
% ========================================================================== %

\code{fancyvrb}  offers  several  advancements   over  the  standard  \LaTeX{}
verbatim environment. Some of them are:

\begin{itemize}\firmlist
    \item
        Footnotes can contain verbatim content.
    \item
        Several  different  verbatim  environments  are provided,  and  it  is
        possible to create new environments.
    \item
        Verbatim content can be stored and recalled.
    \item
        Files can be read and written in verbatim mode.
\end{itemize}

Putting  verbatim  content  in  footnotes  is  accomplished  by  invoking  the
\mintinline{text}{\VerbatimFootnotes}  command  after  the  preamble  at  some
point, after which verbatim content can be put inside footnotes:


% -------------------------------------------------------------------------- %
\section{Footnotes}
\label{sec:fancyvrb:footnotes}
% -------------------------------------------------------------------------- %

\VerbatimFootnotes
After    invoking    VerbatimFootnotes,    verbatim   can    be    put    into
footnotes\footnotemark. Note that the verbatim content in the footnote must be
on a  single line (though the  footnote text itself can  break across multiple
lines of source code.)

\footnotetext{\verb+_And here we are, down in the footnotes!!_+ This is outside
the verbatim content in the footnote.} 

\begin{tcblisting}{listing only}
    \VerbatimFootnotes
    And then we have some text\footnotemark with a footnote, and that footnote
    shall contain verbatim content.
    \footnotetext{\verb+_And here  we are,  down in the  footnotes!!_+ This is
    outside the verbatim content in the footnote.}
\end{tcblisting}


% -------------------------------------------------------------------------- %
\section{Inline Verbatim Content}
\label{sec:fancyvrb:inline}
% -------------------------------------------------------------------------- %

\code{fancyvrb}  allows  convenient  inlining  of  verbatim  content  via  the
\mintinline{text}{\DefineShortVerb}, the \mintinline{text}{\UnDefineShortVerb}
and the \mintinline{text}{\Verb} commands\footnotemark:

\footnotetext{%
    Note that  the underscores in this  example are not necessary,  but merely
    serve to emphasize  that we are in verbatim mode,  because in regular text
    mode it  is a special  character which would need  to be escaped  like so:
    \mintinline{text}{\_}.}

\begin{tcblisting}{listing side text}
    We can write \Verb+_inline verbatim_+
    by defining delimiting characters ad
    hoc, or via:
    \DefineShortVerb{\|}
    And now we  can use 
    |_this_ _delimiter_|  around  inline
    verbatim content,  until its special
    meaning is revoked via
    \UndefineShortVerb{\|}.
    And  now the  |delimiter| no  longer
    has any effect. A new one could also
    be defined.
\end{tcblisting}

% -------------------------------------------------------------------------- %
\section{Verbatim Environments}
\label{sec:fancyvrb:environments}
% -------------------------------------------------------------------------- %

\code{fancyvrb}    provides    several   verbatim    environments. The    most
straightforward replacement for the stock \LaTeX{} \code{verbatim} environment
is the \code{Verbatim} environment\footnotemark:

\footnotetext{%
    For   a  complete   list  of   options,   consult  the   user  manual   at
    \code{fancyvrb}.}

\begin{tcblisting}{listing side text}
    \begin{Verbatim}
    #include <stdio.h>

    main() {
        print("Hello, world!\n");
    }
    \end{Verbatim}
\end{tcblisting}

An example with a lot of options (not all of which might be sensible):

\begin{tcblisting}{listing side text}
    \begin{Verbatim}[%
        gobble=8,
        numbers=left,
        numbersep=4pt,
        showspaces=true,
        showtabs=true,
        baselinestretch=1.5,
        tabsize=4,
        frame=single,
        framerule=1mm,
        framesep=1em,
        rulecolor=\color{teal},
        label={%
            [\fbox{\Large{Hello World}}]%
            end of code}%
        ]
        #include <stdio.h>

        main() {
        	print("Hello, world!\n");
        }
        \end{Verbatim}
\end{tcblisting}

We can also define custom environments:

\begin{tcblisting}{listing side text}
    \DefineVerbatimEnvironment%
        {CustomVerbatim}{Verbatim}{%
        gobble=8,
        numbers=left,
        numbersep=4pt,
        baselinestretch=1.1,
        frame=single,
        framerule=1mm,
        framesep=1em,
        rulecolor=\color{teal}}

    \begin{CustomVerbatim}[%
        label={%
        [\fbox{\Large{Hello World In C}}]
        end of code}]
        #include <stdio.h>

        main() {
            print("Hello, world!\n");
        }
    \end{CustomVerbatim}

    \begin{CustomVerbatim}[%
      label={%
      [\fbox{\Large{Hello World in Perl}}]
      end of code}]
        #!/usr/bin/env perl
        use strict;
        use warnings;
        use v5.10;

        say("Hello, world!");
    \end{CustomVerbatim}
\end{tcblisting}

There is also the possibility to  store and recall verbatim content, which, as
in  the case  of \code{verbatimbox}  package, allows  to use  verbatim content
where it is otherwise not allowed, and to re-use it repeatedly in a convenient
way.   Additionally,  content  can  be  written  to  and  read  from  external
files. Lastly,  it shall  be mentioned  that the  \code{fancyvrb} package  can
interface with  the \code{listings} package (see  Chapter \ref{chap:listings})
for code formatting.


% ========================================================================== %
\newpage
\chapter{The \code{listing} Package \cite{listing}}
\label{chap:listing}
% ========================================================================== %


The  \code{listing}  package is  not  itself  made  for formatting  code,  but
provides  a  new  environment  \code{listing}  similar  to  \code{figure}  and
\code{table}, although not floating, which  can be captioned and indexed. 

Note that other packages provide this  functionality as well, and there may be
clashes  (for  example  with  the \code{minted}  package,  which  provides  an
identically named environment and index list)\footnotemark.

\footnotetext{%
    If you have  a look at the code  of this document, you will  see that some
    workaround magic has been applied to the source code of the \code{listing}
    package to produce a customized version, renaming some crucial commands so
    that they do not clash with the other packages which have been loaded.}

A piece of  example code, using the \code{fancyvrb}  package for formatting
the code, might look like this\footnotemark:

\footnotetext{%
    The   \code{GlobalCustomVerbatim}   environment    is   similar   to   the
    \code{CustomVerbatim}  environment defined  above. The curious  reader may
    feel inclined to look at this document's source code.}

\begin{tcblisting}{listing only,title=Code}
    \begin{listing}
        \begin{GlobalCustomVerbatim}[label={\fbox{\Large{Hello World in Python}}}]
            #!/usr/bin/env python3

            print("Hello, world!")
        \end{GlobalCustomVerbatim}
    \caption{This is a listing with a caption.}
    \end{listing}
\end{tcblisting}

Which will yield as its output\footnotemark:

\footnotetext{%
    The astute reader will  notice that, for once, the output  of this code is
    not  put inside  one  of those  fancy boxes  with  round corners. This  is
    because the  \code{listing} environment leaves what  is called \emph{outer
    par mode}, and thus it will refuse to work inside one of those boxes.}

\begin{mylisting}
    \begin{GlobalCustomVerbatim}[%
    label={\fbox{\Large{Hello World in Python}}}]
        #!/usr/bin/env python3

        print("Hello, world!")
    \end{GlobalCustomVerbatim}
    \caption{This is a listing with a caption.}
\end{mylisting}

The List of Listings can be generated at any place in the document by:

\begin{tcblisting}{listing side text}
\listoflistings
\end{tcblisting}


% ========================================================================== %
\newpage
\chapter{The \code{listings}      Package \cite{listings}}
\label{chap:listings}
% ========================================================================== %

The \code{listings}  package is a  very powerful package for  formatting code,
and probably one of the  more popular choices\footnotemark. It supports inline
code, code  segments in  paragraphs and  floats and  code input  from external
files. Covering the entirety of its capabilities would be far beyond the scope
of this  document; we  shall focus on  a few concise  examples for  common use
cases.  For  more information, as  you may  have guessed, consult  the package
documentation at \cite{listings}.

\footnotetext{Going by my very scientifically reliable gut feeling.}


\begin{tcblisting}{listing side text}
\lstset{language=C}
While writing normal text, code
can be inlined with the listings
package: \lstinline!int i = 5;!.
\end{tcblisting}


Without doing much, the output will look like this:
\begin{tcblisting}{listing side text}
    \begin{lstlisting}[%
        language=C,
        gobble=8]
        #include <stdio.h>

        /* Comment required. */

        main() {
            print("Hello, world!\n");
            int a;
            a = 3;
            if (a = 2) {
                magic(a);
            } else {
                darkmagic(--a);
            }
        }
    \end{lstlisting}
\end{tcblisting}

\newpage
But \code{listings} allows a lot of customization:

\begin{tcblisting}{listing side text}
    \lstdefinestyle{myC}{
        language          = C,
        showstringspaces  = false,
        basicstyle        = \ttfamily,
        numbers           = left,
        numberstyle       = \tiny,
        numbersep         = 4pt,
        commentstyle      = \itshape,
        keywordstyle      = \color{blue},
        stringstyle    = \color{magenta},
        identifierstyle = \color{violet},
        breaklines        = true,
        breakatwhitespace = true,
        tabsize           = 4
    }
    \begin{lstlisting}[%
        language=C,
        style=myC,
        gobble=8]
        #include <stdio.h>

        /* Comment required. */

        main() {
            print("Hello, world!\n");
            int a;
            a = 3;
            if (a = 2) {
                magic(a);
            } else {
                darkmagic(--a);
            }
        }
    \end{lstlisting}
\end{tcblisting}

Note that we  only had to define  \code{keywordstyle}, \code{commentstyle} and
\code{stringstyle} here,  and \code{listings} automatically chose  which words
to highlight in  which manner. A list of supported languages  is in the manual
at \cite{listings}; additional languages and new styles for existing languages
(as done here) can be defined if they  do not exist or if the existing ones do
not  suit your  tastes. Adding new  languages  will obviously  require you  to
define  the  appropriate key  words  and  the way  in  which  they are  to  be
highlighted.

Including external files is  done with the \verb|\lstinputlisting| command. We
can also  give a caption  (regardless of  whether or not  the listing is  in a
float) or, if we don't want  the \textbf{Listing} text before the description,
a \code{title} argument. Putting  a caption also puts the listing  in the list
of listings (invoked by the \verb|\lstlistoflistings| command).

\begin{tcblisting}{}
    \lstinputlisting[
        caption = The \emph{Hello World} program in C,
        frame=lines,
        language=C,
        style=myC]{code/hw.c}
\end{tcblisting}

A particularly nice  feature for didactic purposes  is the \code{escapeinside}
argument, which allows to escape to \LaTeX{} code inside an \code{lstlistings}
environment.

\begin{tcblisting}{}
    \begin{lstlisting}[%
        escapeinside=||,
        basicstyle = \ttfamily,
        gobble=8,
        frame=single,
        frameround=tttt,
        title = Example of \texttt{escapeinside}]
        By using the escapeinside argument, we can |\textcolor{red}{highlight}| 
        particular sections of a code fragment with |\LaTeX| and 
        use arbitrary |\LaTeX| commands in a listing.
    \end{lstlisting}
\end{tcblisting}

Listings  can also  be made  floating by  adding the  \code{float} key  to its
optional arguments.


% ========================================================================== %
\newpage
\chapter{The \code{listingsutf8}  Package \cite{listingsutf8}}
\label{chap:listingsutf8}
% ========================================================================== %


% ========================================================================== %
\chapter{The \code{mlpretty}      Package \cite{mlpretty}}
\label{chap:mlpretty}
% ========================================================================== %


% ========================================================================== %
\chapter{The \code{minted}        Package \cite{minted}}
\label{chap:minted}
% ========================================================================== %


% ========================================================================== %
\chapter{The \code{verbments}        Package \cite{verbments}}
\label{chap:minted}
% ========================================================================== %


% ========================================================================== %
\vfill
\begin{thebibliography}{1}
% ========================================================================== %

    \bibitem{verbatim}
        Rainer Sch\"opf and The \LaTeX{} Team.
        ``\emph{verbatim -– Reimplementation of and extensions to \LaTeX{} verbatim}'',
        Version 1.5q,
        2001-MAR-12.
        [Online],
        \href{http://ctan.org/pkg/verbatim}{\nolinkurl{http://ctan.org/pkg/verbatim}},
        [Accessed: 2017-MAR-22].

    \bibitem{verbatimbox}
        Steven B. Segletes.
        ``\emph{verbatimbox – Deposit verbatim text in a box}'',
        Version 3.13,
        2014-MAR-12.
        [Online],
        \href{http://ctan.org/pkg/verbatimbox}{\nolinkurl{http://ctan.org/pkg/verbatimbox}},
        [Accessed: 2017-MAR-22].

    \bibitem{fancyvrb}
        Timothy Van Zandt and Herbert Vo\ss{} and Denis Girou.
        ``\emph{fancyrvb -- Sophisticated verbatim text}'',
        Version 2.8,
        2010-MAY-15.
        [Online],
        \href{http://ctan.org/pkg/fancyvrb}{\nolinkurl{http://ctan.org/pkg/fancyvrb}},
        [Accessed: 2017-MAR-22].

    \bibitem{listing}
        Volker Kuhlmann and Matthew Hebley.
        ``\emph{listing -- Produce formatted program listings}'',
        Version 1.2,
        1999-MAY-25.
        [Online],
        \href{http://ctan.org/pkg/listing}{\nolinkurl{http://ctan.org/pkg/listing}},
        [Accessed: 2017-MAR-22].

    \bibitem{listings}
        Brooks Moses and Carsten Heinz and Jobst Hoffman.
        ``\emph{listings -- Typeset source code listings using \LaTeX}'',
        Version 1.6,
        2015-JUN-04.
        [Online],
        \href{http://ctan.org/pkg/listings}{\nolinkurl{http://ctan.org/pkg/listings}},
        [Accessed: 2017-MAR-22].

    \bibitem{listingsutf8}
        Heiko Oberdiek.
        ``\emph{listingsutf8 -- Allow UTF-8 in listings input}'',
        Version 1.3,
        2016-MAY-16.
        [Online],
        \href{http://ctan.org/pkg/listingsutf8}{\nolinkurl{http://ctan.org/pkg/listingsutf8}},
        [Accessed: 2017-MAR-22].

    \bibitem{mlpretty}
        Julien Cretel.
        ``\emph{matlab-prettifier -- Pretty-print Matlab source code}'',
        Version 0.3,
        2014-JUN-19.
        [Online],
        \href{http://ctan.org/pkg/matlab-prettifier}{\nolinkurl{http://ctan.org/pkg/matlab-prettifier}},
        [Accessed: 2017-MAR-22].

    \bibitem{minted}
        Konrad Rudolph and Geoffrey Poore.
        ``\emph{minted -- Highlighted source code for \LaTeX}'',
        Version 2.4.1,
        2016-OCT-31.
        [Online],
        \href{http://ctan.org/pkg/minted}{\nolinkurl{http://ctan.org/pkg/minted}},
        [Accessed: 2017-MAR-22].

    \bibitem{verbments}
        Dejan Živkovic.
        ``\emph{verbments -- Syntax highlighting of source code in \LaTeX{} documents}'',
        Version 1.2,
        2011-AUG-20.
        [Online],
        \href{http://ctan.org/pkg/verbments}{\nolinkurl{http://ctan.org/pkg/verbments}},
        [Accessed: 2017-MAR-22].

    \bibitem{ctan:topic:listings}
        Comprehensive \TeX{} Archive Network.
        ``\emph{Topic listing}''.
        [Online],
        \href{http://ctan.org/topic/listing}{\nolinkurl{http://ctan.org/topic/listing}},
        [Accessed: 2017-MAR-22].

    \bibitem{ctan:topic:verbatim}
        Comprehensive \TeX{} Archive Network.
        ``\emph{Topic verbatim}''.
        [Online],
        \href{http://ctan.org/topic/verbatim}{\nolinkurl{http://ctan.org/topic/verbatim}},
        [Accessed: 2017-MAR-22].


\end{thebibliography}

\end{document}
