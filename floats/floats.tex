\documentclass[article,a4paper,oneside,10pt]{memoir}

% -------------------------------------------------------------------------- %
% Page Layout                                                                %
% -------------------------------------------------------------------------- %
\setlrmarginsandblock{0.142857111\paperwidth}{0.142857111\paperwidth}{*}
\setulmarginsandblock{0.111111111\paperheight}{*}{1.25}
\checkandfixthelayout
\newfixedcaption{\figcaption}{figure}
\newfixedcaption{\tabcaption}{table}
\captiontitlefont{\small}
\captionnamefont{\bfseries\small}
\captiondelim{: }

% -------------------------------------------------------------------------- %
% Bibliography                                                               %
% -------------------------------------------------------------------------- %
\renewcommand{\bibsection}{%
    \chapter{References}%
    \prebibhook}
\bibintoc
\renewcommand{\cftdot}{\textcolor{gray}{.}}
\renewcommand{\cftchapterdotsep}{\cftdotsep} % Chapters have dots in ToC
\renewcommand{\cftsectiondotsep}{\cftnodots} % Sections have no dots in ToC


% -------------------------------------------------------------------------- %
% Chapter and Section Headings                                               %
% -------------------------------------------------------------------------- %
\makeatletter
\makechapterstyle{alpensec}{%
  \chapterstyle{default}
  \setlength{\beforechapskip}{3.5ex \@plus 1ex \@minus .2ex}
  \renewcommand*{\chapterheadstart}{\vspace{\beforechapskip}}
  \setlength{\afterchapskip}{2.3ex \@plus .2ex}
  \renewcommand{\printchaptername}{}
  \renewcommand{\chapternamenum}{}
  \renewcommand{\chaptitlefont}{\sffamily\LARGE\bfseries}
  \renewcommand{\chapnumfont}{\chaptitlefont}
  \renewcommand{\printchapternum}{\chapnumfont \thechapter\quad}
  \renewcommand{\afterchapternum}{}}

\renewcommand{\section}{%
  \sechook%
  \@startsection{section}{1}%  level 1
      {\secindent}%            heading indent
      {\beforesecskip}%        skip before the heading
      {\aftersecskip}%         skip after the heading
      {\sffamily\secheadstyle}} % font

\makeatother

\chapterstyle{alpensec}


% -------------------------------------------------------------------------- %
% Packages                                                                   %
% -------------------------------------------------------------------------- %
\usepackage[%
        bookmarksnumbered=true,
        colorlinks=true,
        linkcolor=cyan!50!blue,
        citecolor=violet,
        urlcolor=purple,
    ]{hyperref}
\usepackage[light,nott]{kpfonts}
\usepackage{microtype}
\usepackage[english]{babel}
\usepackage[T1]{fontenc}
\usepackage[utf8]{inputenc}
\usepackage{xcolor-solarized}
\usepackage{lipsum}
\usepackage{graphicx}
\usepackage{minted}
\renewcommand{\listingscaption}{\small\bfseries Listing}


% -------------------------------------------------------------------------- %
% Macros                                                                     %
% -------------------------------------------------------------------------- %
\newcommand\code[1]{\texttt{#1}}


% -------------------------------------------------------------------------- %
% Title                                                                      %
% -------------------------------------------------------------------------- %
\title{\textsf{\Huge  The Black Magic of Floats in \LaTeX}}
\author{Raphael Frey\\[2mm]\small%
    \href{https://github.com/alpenwasser/TeX/tree/master/floats}
         {\nolinkurl{https://github.com/alpenwasser/TeX/}}}

\date{\vspace{1em}\today}

% ************************************************************************** %
\begin{document}
% ************************************************************************** %


\maketitle

\begin{abstract}
    The behavior  of floats can  often be  confusing for the  uninitiated, and
    yield  unexpected results. This  document gives  a brief  overview on  the
    subject, primarily based on Leslie  Lamport's \emph{\LaTeX{} -- A Document
    Preparation System} \cite{lamport}.

    I will not cover every possible edge case, but present some usage examples
    and common problem one tends to run into while working with floats.
\end{abstract}

\tableofcontents*
\listoflistings
\listoffigures
\listoftables


% ========================================================================== %
\newpage
\chapter{What Are Floats, Anyway?}
\label{chap:what-are-floats}
% ========================================================================== %

Normal text is broken by \TeX{} across lines and pages automatically. For some
content, such as images, are not  well-suited to being split into pieces. That
is what floating environments are for: To provide a way to put such content in
a place where it does not need to be broken; a way for it to \emph{float} to a
suitable location for an optimal overall result.

Two  such  floating environments  are  provided  by \LaTeX{}  by  default: The
\verb|figure|  and  the  \verb|table| environment. There  are  packages  which
define more  floating environments (for example,  the \verb|listings| packages
can let its code listings float, if  so desired), or the user may define their
own floating environments, if they so wish.

Fundamentally, the only important difference between these environments is how
they  are  captioned  and  numbered: \verb|figure|  environments  will  get  a
different caption and  number to a \verb|table|  environment. However, one may
in principle put pretty much  anything one desires into either environment. As
long  as  the  code  itself  is  valid,  \LaTeX{}  will  not  complain. Figure
\ref{fig:lipsum} and Table \ref{lipsum} demonstrate this by placing some Lorem
Ipsum text inside their environments.

The floating behavior can be demonstrated by the fact that, in the source code
of this document, \ref{fig:lipsum} and \ref{tab:lipsum} are placed right after
this sentence. In the resulting document, they may be placed wherever \LaTeX{}
deems most suitable (probably at the top of this page\footnotemark).

\footnotetext{%
    \LaTeX{} usually tries  to place floats either  at the top of  a page, the
    bottom of a page, or on a separate page. See \ref{chap:placement} for more
    information.}

\begin{figure}
    \small\lipsum[2]
    \caption{A \texttt{figure} environment with placeholder text}
    \label{fig:lipsum}
\end{figure}

\begin{table}
    \caption{A \texttt{table} environment with placeholder text}
    \label{tab:lipsum}
    \small\lipsum[2]
\end{table}

Because  captions are  a moving  argument (Section  3.5.1 in  \cite{lamport}),
fragile  commands  such  as  linebreaks   inside  them  must  be  preceded  by
\verb|\protect|, as shown in the caption of Figure \ref{fig:protect}.

\begin{figure}
    \centering
    \includegraphics[height=3cm,width=4.5cm]{images/grid8cm.png}
    \caption[Linebreaks in Captions]{%
        One can  place arbitrary  content inside a  \code{figure} environment,
        though this does not usually make much sense.\protect\\
        Note      that       you      can      also       have      linebreaks
        inside     a      \code{caption},     but     it      requires     the
        \code{\textbackslash{}protect} command before  the linebreak, like so:
        \code{\textbackslash{}protect\textbackslash\textbackslash}}
    \label{fig:protect}
\end{figure}

The \verb|\caption| command can only be  used inside a floating environment by
default.   If  you require  captions  for  non-floating arguments,  there  are
packages which provide such facilities,  as well as more caption customisation
options,   see   \cite{ctan:package:caption,ctan:topic:caption}  and   Chapter
\ref{chap:alternatives} of this document.


% ========================================================================== %
\chapter{Basic Usage}
\label{chap:basic-usage}
% ========================================================================== %

Listing \ref{lst:figure}\footnotemark  shows the  basic code for  including an
external graphics file inside a \verb|figure| environment.

\footnotetext{%
    Incidentally, Listing \ref{lst:figure}  is one of those cases  where a new
    type  of floating  environment  has been  provided; in  this  case by  the
    \texttt{minted} package.}

\begin{listing}
    \begin{minted}[autogobble]{tex}
        \begin{figure}
            \includegraphics[height=3cm,width=4.5cm]{images/grid8cm.png}
            \caption{A very interesting and slightly distorted picture of a grid}
            \label{fig:distorted-grid}
        \end{figure}
    \end{minted}
    \caption{Code block for includeing a graphics in a figure}
    \label{lst:figure}
\end{listing}

It is often desirable  to center a table or a picture, in  which case we add a
\verb|\centering| directive into the environment:

\cite{stackexch:center-centering}
\begin{listing}
    \begin{minted}[autogobble,escapeinside=||]{text}
        \begin{table}
            |\textcolor{red}{\textbf{\textbackslash{}centering}}|
            \begin{tabular}{ll}
                \toprule
                Experiment Input  & Experiment Output        \\
                \midrule
                interesting thing & interesting result!      \\
                boring thing      & mildly surprising result \\
                weird thing       & very unexpected result   \\
                fascinating thing & machine broke            \\
                Xenomorph XX121   & very dead scientists     \\
                \bottomrule
            \end{tabular}
            \caption{Results for an experiment}
            \label{tab:experiment}
        \end{table}
    \end{minted}
    \caption{%
        Centering  a  \texttt{tabular}   environment  inside  a \texttt{table}
        floating environment}
    \label{lst:figure}
\end{listing}

%\begin{figure}
%    \includegraphics[width=\textwidth]{images/aw--fluffy--2014-12-28--21.jpeg}
%    \caption{A picture of our canine in snowy conditions}
%    \label{fig:fluffy}
%\end{figure}
%
%\begin{figure}
%    \includegraphics[width=\textwidth]{images/aw--helios--2016-02-24--01.jpeg}
%    \caption{My box of computing}
%    \label{fig:helios}
%\end{figure}

%\begin{figure}
%    \lipsum[2]
%    \caption{%
%        One can  place arbitrary  content inside a  \code{figure} environment,
%        though this does not usually make much sense.\protect\\
%        Note      that       you      can      also       have      linebreaks
%        inside     a      \code{caption},     but     it      requires     the
%        \code{\textbackslash{}protect} command before  the linebreak, like so:
%        \code{\textbackslash{}protect\textbackslash\textbackslash}}
%    \caption{fig:lipsum}
%\end{figure}


% ========================================================================== %
\chapter{Placement Options}
\label{chap:placement}
% ========================================================================== %

Only use these towards the end of writing.

% ========================================================================== %
\chapter{Help, My Floats Are Jinxed!}
\label{chap:jinxed}
% ========================================================================== %

Too many floats, not enough text, \ldots


% ========================================================================== %
\chapter{\LaTeX's Dark Magic}
\label{chap:innards}
% ========================================================================== %


% ========================================================================== %
\chapter{Alternatives to Using Floats}
\label{chap:alternatives}
% ========================================================================== %


% ========================================================================== %
\newpage
\begin{thebibliography}{1}
% ========================================================================== %

    \bibitem{lamport}
        Leslie Lamport, Digital Equipment Corporation,
        ``\emph{\LaTeX{} -- A Document Preparation System}'',
        2nd Edition,
        1994,
        Addison-Wesley Publishing Company.

    \bibitem{ctan:package:caption}
        Comprehensive \TeX{} Archive Network.
        ``\emph{Package caption -- Customising captions in floating environments}''.
        [Online],
        \href{http://ctan.org/pkg/caption}{\nolinkurl{http://ctan.org/pkg/caption}},
        [Accessed: 2017-MAR-26].

    \bibitem{ctan:topic:caption}
        Comprehensive \TeX{} Archive Network.
        ``\emph{Topic caption}''.
        [Online],
        \href{http://ctan.org/topic/caption}{\nolinkurl{http://ctan.org/topic/caption}},
        [Accessed: 2017-MAR-26].

    \bibitem{stackexch:center-centering}
        Enrico Gregorio,
        ``\emph{When should we use \texttt{\textbackslash{}begin\{center\}} 
        instead of \texttt{\textbackslash{}centering?}}'',
        [Online],
        \href{http://tex.stackexchange.com/a/23653}
             {\nolinkurl{http://tex.stackexchange.com/a/23653}},
        [Accessed: 2017-MAR-26].

\end{thebibliography}

\end{document}
