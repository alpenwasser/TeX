\documentclass[a4paper,11pt]{article}
\usepackage{lipsum}

%% Use these two together...
%\usepackage[numbers]{natbib}
%\bibliographystyle{IEEEtranN}

% ...or this one
\bibliographystyle{officialIEEEtran}

\usepackage[nottoc]{tocbibind}

\begin{document}

This bibliography was generated using the IEEEtran.bst bibliography style file
from IEEE themselves, downloaded on 2017-APR-09:

\noindent\texttt{http://www.ieee.org/documents/IEEEtranBST.zip}

%You  will only  be happy  once it's  incomprehensible, illogical  and actively
%hostile to the user.
%
%IEEE  citations sind  ein  hervorragendes beispiel,  wie  eine Gruppe  Leuten,
%von  denen  vermutlich(/hoffentlich?)  nur   wenige  dumm  sind,  durch  einen
%kooperativen Effort  eine absolut(/ueberraschend?) idiotische  und laecherlich
%komplizierte Loesung fuer ein  relativ(/eigentlich?) simples Problem entwerfen
%koennen. Nicht das einzige Beispiel, aber ein hervorragendes.

\nocite{*}
%\cite{fancy-article},
%\cite{funny-book},
%\cite{funny-inbook},
%\cite{funny-incollection},
%\cite{serious-booklet},
%\cite{useless-manual},
%\cite{boring-conference},
%\cite{acceptable-proceedings},
%\cite{splendid-mastersthesis},
%\cite{splendid-phdthesis},
%\cite{innovative-report},
%\cite{elec-resource},
%\cite{patent},
%\cite{periodical},
%\cite{the-standard},
%\cite{misc-things},

\bibliography{references}
\end{document}
