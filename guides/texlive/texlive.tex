\documentclass[a4paper,11pt]{article}
\usepackage[left=30mm,right=30mm,top=30mm,bottom=30mm]{geometry}
\usepackage[ngerman]{babel}
\usepackage{kpfonts}
\usepackage{graphicx}
\usepackage{hyperref}
\renewcommand{\familydefault}{\sfdefault}
\def\tl{\TeX{} Live}
\newcommand{\screenshot}[1]{%
    \noindent\begin{minipage}{\textwidth}
        \vspace{1em}
        \centering
        \includegraphics[width=0.8\textwidth]{images/#1}
        \vspace{1em}
    \end{minipage}}
\title{\tl{} Installation Guide}
\author{Raphael Frey}
\date{\today}

\begin{document}
\maketitle

Der Installer (\texttt{install-tl-windows.exe}) kann heruntergeladen werden von \\
\href{https://www.tug.org/texlive/acquire-netinstall.html}
     {https://www.tug.org/texlive/acquire-netinstall.html}

Der  Installer  beinhaltet   keine  Installationsdateien. Hat  man  gen\"ugend
Speicherplatz  auf  dem Rechner  (etwa  5  Gigabyte),  empfielt es  sich,  die
volle Installation auszuf\"uhren  (``Simple install (big)''). Die Installation
erfordert keine grosse Konfiguration.

\vspace{1em}
Installer starten\ldots

\screenshot{texlive00.png}

\newpage
Brav durchklichen\ldots

\screenshot{texlive01.png}
\screenshot{texlive02.png}

\newpage

Anschliessend sollte das Setup-Programm starten:

\screenshot{texlive03.png}
\screenshot{texlive04.png}
\screenshot{texlive05.png}
\screenshot{texlive06.png}
\screenshot{texlive07.png}

Der n\"achste Schritt dauert eine Weile. Es empfielt sich eine Kaffeepause\footnotemark

\footnotetext{%
    Oder was auch immer der geneigte Benutzer/die geneigte Benutzering bevorzugt.%
}

\screenshot{texlive08.png}
\screenshot{texlive09.png}
\screenshot{texlive10.png}

\newpage
Nach  der  Installation sollte  einige  neue  Programme  auf dem  Rechner  zur
Verf\"ugung stehen. Der  \emph{TeX Live Manager} dient  zur Administration der
installierten (und noch nicht installierten) Packages. Es lassen sich Packages
installieren, deinstallieren und aktualisieren.

\screenshot{texlive13.png}

Der     standardm\"assig     Installierte      Editor     ist     \TeX{}works:

\screenshot{texworks0.png}

Die Installation wurde ausgef\"uhrt auf einem Windows 8.1-System.
\end{document}
