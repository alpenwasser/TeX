% ******************************************************** %
% DOCUMENT INFORMATION                                     %
% ******************************************************** %
%                                                          %
% Purpose of Document                                      %
% -------------------                                      %
% Some Specific Remarks on LaTeX                           %
%                                                          %
% Institution                                              %
% -----------                                              %
% University of Applied Sciences and Arts Northwestern     %
% Switzerland, School of Engineering                       %
%                                                          %
% Degree Program                                           %
% --------------                                           %
% Electrical Engineering and Information Technology, BSc.  %
%                                                          %
% Author & Copyright                                       %
% ------------------                                       %
% Raphael Frey, raphael.frey@students.fhnw.ch              %
%               rmfrey@runbox.com                          %
%                                                          %
% Date: 2017-APR-26                                        %
% ******************************************************** %

\documentclass{beamer}                % Presentation Version
%\documentclass[trans]{beamer}         % Transparency Version
%\documentclass[handout]{beamer}       % Handout Version

% Beamer Setup --------------------------------------------------------------- %
%\addtobeamertemplate{background canvas}{\transdissolve[duration=1]\hspace{-0.29em}}{}
%\usetheme[titleformat=smallcaps,numbering=none]{metropolis}
\usetheme[titleformat=smallcaps,progressbar=frametitle]{metropolis}

% Packages ------------------------------------------------------------------- %
\usepackage[ngerman]{babel}
\usepackage{xcolor-solarized}
\usetikzlibrary{calc}
\usetikzlibrary{positioning}
\usetikzlibrary{arrows}
\usetikzlibrary{calc}
\usetikzlibrary{fit}
\usetikzlibrary{spy}
\usetikzlibrary{backgrounds}
\usetikzlibrary{shapes.symbols}
\usepackage{hyperref}
\usepackage{dirtree}
\usepackage{minted}
\setminted{%
    %style=xcode,
    %style=trac,
    %style=paraiso-light,
    %style=lovelace,
    style=murphy,
    bgcolor=solarized-base3,
    linenos=true,
}
%\setbeamercolor{alerted text}{fg=solarized-red}


% Macros --------------------------------------------------------------------- %
% Arrow for use in text
\def\tikzrarrow{%
    \tikz[baseline=-0.67ex]\draw[double,very thick,-{stealth}](0,0)--(0.5,0);}
\newlength{\yOffs}
\newlength{\xOffs}

% Title Setup ---------------------------------------------------------------- %
\title{\vspace*{4em}\Huge\LaTeX}
\subtitle{\hfill Trouble on the Horizon}
%\date{\today}
\date{26. April 2017}
\author{%
    Raphael Frey%
    \hfill%
    \scriptsize\texttt{%
        \href{mailto:rmfrey@runbox.com}%
        {rmfrey@runbox.com}}}

%\institute{IME}
\titlegraphic{\includegraphics[height=2em]{images/fhnwLogoDE-solarized-base02.eps}}
\subject{LaTeX -- Kurze Einf\"uhrung}
\keywords{LaTeX Introduction Primer FHNW Overview}
%http://web.fhnw.ch/cd/corporate-design/logos-fur-die-hochschulen

% **************************************************************************** %
\begin{document}                                                              
% **************************************************************************** %

\frame[plain]{\titlepage} % -------------------------------------------- FRAME %


%\begin{frame}<handout:0> % --------------------------------------------- FRAME %
%    \frametitle{Programm}
%    \tableofcontents
%\end{frame}


% ---------------------------------------------------------------------------- %
%\section<handout:0>{Was und Warum?} % --------------------------------- FRAME %
% ---------------------------------------------------------------------------- %

\begin{frame} % -------------------------------------------------------- FRAME %
    \frametitle{Lohnenswerte Packages}
    \begin{itemize}
        \item hyperref
        \item microtype
        \item babel/polyglossia
        \item inputenc
        \item fontenc, fontspec, luatextra: https://tex.stackexchange.com/a/44701/131649
    \end{itemize}
\end{frame}

\begin{frame} % -------------------------------------------------------- FRAME %
    \frametitle{Bibliographie}
\end{frame}

\begin{frame} % -------------------------------------------------------- FRAME %
    \frametitle{Modulare Dokumente}
\end{frame}

\begin{frame} % -------------------------------------------------------- FRAME %
    \frametitle{Floats}
\end{frame}

\begin{frame} % -------------------------------------------------------- FRAME %
    \frametitle{Farben}
\end{frame}

\begin{frame} % -------------------------------------------------------- FRAME %
    \frametitle{\texttt{minipages}}
\end{frame}

\begin{frame} % -------------------------------------------------------- FRAME %
    \frametitle{Code Listings}
\end{frame}

\begin{frame} % -------------------------------------------------------- FRAME %
    \frametitle{Externe PDF-Dokumente}
\end{frame}

\begin{frame} % -------------------------------------------------------- FRAME %
    \frametitle{Elektrische Schaltungen zeichnen}
\end{frame}

\begin{frame} % -------------------------------------------------------- FRAME %
    \frametitle{Daten Plotten}
    \begin{itemize}
        \item matlab2tikz
        \item pgfplots
    \end{itemize}
\end{frame}

\begin{frame} % -------------------------------------------------------- FRAME %
    \frametitle{Alternative Klassen}
\end{frame}

\begin{frame} % -------------------------------------------------------- FRAME %
    \frametitle{Support}

    \begin{itemize}
        \item
            e-Mail:
            \texttt{\href{mailto:raphael.frey@students.fhnw.ch}
            {raphael.frey@students.fhnw.ch}}
        \item
            Raum 4.223
        \item
            \alert{\href{https://github.com/alpenwasser/TeX/}
                        {https://github.com/alpenwasser/TeX/}}
        \item
            Weiteres Programm: 26. April
            \begin{itemize}
                \item
                    modulare Dokumentstruktur
                \item
                    Bibliographie
                \item
                    Floats (Tabellen, Bilder, \ldots )
                \item
                    PGF/Ti\emph{k}Z
                %\item
                %    minipages
                %\item
                %    ISO-31
                \item
                    Listings, minted, \ldots
                \item
                    etc. etc. etc.
            \end{itemize}
    \end{itemize}
\end{frame}

% ---------------------------------------------------------------------------- %
\section<handout:0| trans:0>*{Fragen?} % ------------------------------- FRAME %
% ---------------------------------------------------------------------------- %
\end{document}
