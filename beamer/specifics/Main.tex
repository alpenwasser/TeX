% ******************************************************** %
% DOCUMENT INFORMATION                                     %
% ******************************************************** %
%                                                          %
% Purpose of Document                                      %
% -------------------                                      %
% Some Specific Remarks on LaTeX                           %
%                                                          %
% Institution                                              %
% -----------                                              %
% University of Applied Sciences and Arts Northwestern     %
% Switzerland, School of Engineering                       %
%                                                          %
% Degree Program                                           %
% --------------                                           %
% Electrical Engineering and Information Technology, BSc.  %
%                                                          %
% Author & Copyright                                       %
% ------------------                                       %
% Raphael Frey, raphael.frey@students.fhnw.ch              %
%               rmfrey@runbox.com                          %
%                                                          %
% Date: 2017-APR-26                                        %
% ******************************************************** %

\documentclass{beamer}                % Presentation Version
%\documentclass[trans]{beamer}         % Transparency Version
%\documentclass[handout]{beamer}       % Handout Version

% ------------------------------------------------------------ Beamer Setup %<<<
%\addtobeamertemplate{background canvas}{\transdissolve[duration=1]\hspace{-0.29em}}{}
%\usetheme[titleformat=smallcaps,numbering=none]{metropolis}
\usetheme[titleformat=smallcaps,progressbar=frametitle]{metropolis}
%>>>

% ---------------------------------------------------------------- Packages %<<<
\usepackage[ngerman]{babel}
\usepackage{xcolor-solarized}
\usetikzlibrary{calc}
\usetikzlibrary{positioning}
\usetikzlibrary{arrows}
\usetikzlibrary{calc}
\usetikzlibrary{fit}
\usetikzlibrary{spy}
\usetikzlibrary{backgrounds}
\usetikzlibrary{shapes.symbols}
\usepackage{hyperref}
\usepackage{dirtree}
\usepackage{minted}
\setminted{%
    %style=xcode,
    %style=trac,
    %style=paraiso-light,
    %style=lovelace,
    style=murphy,
    bgcolor=solarized-base3,
    linenos=true,
}
%\setbeamercolor{alerted text}{fg=solarized-red}
%>>>


% ------------------------------------------------------------------ Macros %<<<
% Arrow for use in text
\def\tikzrarrow{%
    \tikz[baseline=-0.67ex]\draw[double,very thick,-{stealth}](0,0)--(0.5,0);}
\newlength{\yOffs}
\newlength{\xOffs}
\newcommand*\code[1]{\texttt{#1}}
\newcommand\source[1]{%
    \tikz[remember picture,overlay] 
        \node[align=left,anchor=south west,text=solarized-base1,font=\tiny]
        at (current page.south west)
        {Quelle: #1};
}
%>>>

% ------------------------------------------------------------- Title Setup %<<<
\title{\vspace*{4em}\Huge\LaTeX}
\subtitle{\hfill Trouble on the Horizon}
%\date{\today}
\date{26. April 2017}
\author{%
    Raphael Frey%
    \hfill%
    \scriptsize\texttt{%
        \href{mailto:rmfrey@runbox.com}%
        {rmfrey@runbox.com}}}

%\institute{IME}
\titlegraphic{\includegraphics[height=2em]{images/fhnwLogoDE-solarized-base02.eps}}
\subject{LaTeX -- Kurze Einf\"uhrung}
\keywords{LaTeX Introduction Primer FHNW Overview}
%http://web.fhnw.ch/cd/corporate-design/logos-fur-die-hochschulen
%>>>

% **************************************************************************** %
\begin{document}                                                              
% **************************************************************************** %

\frame[plain]{\titlepage} % -------------------------------------- TITLE FRAME %

\begin{frame} % ------------------------------------- NICE-TO-HAVE PACKAGES %<<<
    \frametitle{Lohnenswerte Packages}
    \begin{itemize}
        \item hyperref
        \item microtype
        \item babel/polyglossia
        \item inputenc
        \item fontenc, fontspec, luatextra: https://tex.stackexchange.com/a/44701/131649
    \end{itemize}
\end{frame}%>>>

\begin{frame} % ------------------------------------------BIBTEX FLOW CHART %<<<
    \frametitle{Bib\TeX}
    \hspace*{-1em}%
    \begin{tikzpicture}[
        draw=solarized-base02,
        text=solarized-base02,
        rounded corners=2mm,
        every node/.append style={
            draw,
            thick,
            fill=solarized-base2,
            % technically not necessary with the global rounded corners option:
            rounded corners=2mm, 
            inner sep=2mm,
            font=\fontsize{9}{10}\selectfont,
        },
        inputfile/.style={
            draw=solarized-orange,
            double,
        },
        signal/.append style={
            text=solarized-base2,
            fill=solarized-magenta,
            rounded corners=1mm,
        },
    ]
        \setlength{\xOffs}{5mm}
        \setlength{\yOffs}{6mm}

        \node[inputfile] (tex) 
            {\code{.tex}};

        \node (latex1)
            [xshift=\xOffs,yshift=-\yOffs,rotate=-45,signal,shape = signal,signal from =west,signal to=east,below right of=tex]
            {latex};

        \node (aux1) 
            [xshift=2*\xOffs,yshift=-\yOffs,right of=latex1]
            {\code{.aux}};

        \node (latex2)
            [xshift=2.5*\xOffs,signal,shape = signal,signal from =west,signal to=east,right of=aux1]
            {latex};

        \node (bibtex)
            [yshift=-2.33 * \yOffs,signal,shape = signal,signal from =west,signal to=east,below of=aux1]
            {bibtex};

        \node (aux2) 
            [xshift=2.5*\xOffs,right of=latex2]
            {\code{.aux}};

        \node (latex3)
            [xshift=2*\xOffs,rotate=-45,signal,shape = signal,signal from =west,signal to=east,right of=aux2]
            {latex};

        \node (bbl)
            [yshift=-\yOffs,below of=latex2]
            {\code{.bbl}};

        \node[inputfile] (bib)
            [xshift=-9* \xOffs,yshift=-\yOffs,below of=latex2]
            {\code{.bib}};

        \node[inputfile] (bst)
            [yshift=-\yOffs,below of=bib]
            {\code{.bst}};

        \node (blg)
            [yshift=-\yOffs,below of=bbl]
            {\code{.blg}};

        \node (pdf) 
            [xshift=8.5 * \xOffs,right of=blg]
            {\color{solarized-red}\code{.pdf,.dvi}};

        \draw[thick,-{stealth}]
            (tex.south) to[out=-90,in=135] (latex1.west);

        \draw[thick,-{stealth}]
            (tex.east) to[out=0,in=180] (latex2.west);

        \draw[thick,-{stealth}]
            (tex.east) to[out=0,in=135] (latex3.west);

        \draw[thick,-{stealth}]
            (latex1.east) to[out=-45,in=180] (aux1.west);

        \draw[thick,-{stealth}]
            (aux1.east) to[out=0,in=180] (latex2.west);

        \draw[thick,-{stealth}]
            (latex2.east) to[out=0,in=180] (aux2.west);

        \draw[thick,-{stealth}]
            (latex3.east) to[out=-45,in=90] (pdf.north);

        \draw[thick,-{stealth}]
            (bib.east) to[out=0,in=180] (bibtex.west);

        \draw[thick,-{stealth}]
            (bst.east) to[out=0,in=180] (bibtex.west);

        \draw[thick,-{stealth}]
            (bibtex.east) to[out=0,in=180] (bbl.west);

        \draw[thick,-{stealth}]
            (bbl.north) -- (latex2.south);

        \draw[thick,-{stealth}]
            (bibtex.east) to[out=0,in=180] (blg.west);

        \draw[thick,-{stealth}]
            (aux2.east) to[out=0,in=135] (latex3.west);

    \end{tikzpicture}

    \source{Mittelbach et al.,\emph{The \LaTeX\ Companion}, 2nd Ed., p.688, Fig. 12.1}
\end{frame}%>>>

%\begin{frame} % -------------------------------------------------------- FRAME %
%    \frametitle{Modulare Dokumente}
%\end{frame}
%
%\begin{frame} % -------------------------------------------------------- FRAME %
%    \frametitle{Floats}
%\end{frame}
%
%\begin{frame} % -------------------------------------------------------- FRAME %
%    \frametitle{Farben}
%\end{frame}
%
%\begin{frame} % -------------------------------------------------------- FRAME %
%    \frametitle{\texttt{minipages}}
%\end{frame}
%
%\begin{frame} % -------------------------------------------------------- FRAME %
%    \frametitle{Code Listings}
%\end{frame}
%
%\begin{frame} % -------------------------------------------------------- FRAME %
%    \frametitle{Externe PDF-Dokumente}
%\end{frame}
%
%\begin{frame} % -------------------------------------------------------- FRAME %
%    \frametitle{Elektrische Schaltungen zeichnen}
%\end{frame}
%
%\begin{frame} % -------------------------------------------------------- FRAME %
%    \frametitle{Daten Plotten}
%    \begin{itemize}
%        \item matlab2tikz
%        \item pgfplots
%    \end{itemize}
%\end{frame}
%
%\begin{frame} % -------------------------------------------------------- FRAME %
%    \frametitle{Alternative Klassen}
%\end{frame}
%
%\begin{frame} % -------------------------------------------------------- FRAME %
%    \frametitle{Support}
%
%    \begin{itemize}
%        \item
%            e-Mail:
%            \texttt{\href{mailto:raphael.frey@students.fhnw.ch}
%            {raphael.frey@students.fhnw.ch}}
%        \item
%            Raum 4.223
%        \item
%            \alert{\href{https://github.com/alpenwasser/TeX/}
%                        {https://github.com/alpenwasser/TeX/}}
%        \item
%            Weiteres Programm: 26. April
%            \begin{itemize}
%                \item
%                    modulare Dokumentstruktur
%                \item
%                    Bibliographie
%                \item
%                    Floats (Tabellen, Bilder, \ldots )
%                \item
%                    PGF/Ti\emph{k}Z
%                %\item
%                %    minipages
%                %\item
%                %    ISO-31
%                \item
%                    Listings, minted, \ldots
%                \item
%                    etc. etc. etc.
%            \end{itemize}
%    \end{itemize}
%\end{frame}

% ---------------------------------------------------------------------------- %
\section<handout:0| trans:0>*{Fragen?} % ------------------------------- FRAME %
% ---------------------------------------------------------------------------- %
\end{document}
% vim: foldenable foldcolumn=4 foldmethod=marker foldmarker=<<<,>>>
