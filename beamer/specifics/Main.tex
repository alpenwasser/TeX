% **************************************************** <<< %
% DOCUMENT INFORMATION                                     %
% ******************************************************** %
%                                                          %
% Purpose of Document                                      %
% -------------------                                      %
% Some Specific Remarks on LaTeX                           %
%                                                          %
% Institution                                              %
% -----------                                              %
% University of Applied Sciences and Arts Northwestern     %
% Switzerland, School of Engineering                       %
%                                                          %
% Degree Program                                           %
% --------------                                           %
% Electrical Engineering and Information Technology, BSc.  %
%                                                          %
% Author & Copyright                                       %
% ------------------                                       %
% Raphael Frey, raphael.frey@students.fhnw.ch              %
%               rmfrey@runbox.com                          %
%                                                          %
% Date: 2017-APR-26                                        %
% **************************************************** >>> %

% ================================================================= PREAMBLE <<<
% ----------------------------------------------------------- DOCUMENT CLASS <<<
\documentclass{beamer}                % Presentation Version
%\documentclass[trans]{beamer}         % Transparency Version
%\documentclass[handout]{beamer}       % Handout Version
%>>>
% ------------------------------------------------------------- Beamer Setup <<<
%\addtobeamertemplate{background canvas}{\transdissolve[duration=1]\hspace{-0.29em}}{}
%\usetheme[titleformat=smallcaps,numbering=none]{metropolis}
\usetheme[titleformat=smallcaps,progressbar=frametitle]{metropolis}
%>>>
% ----------------------------------------------------------------- Packages <<<
\usepackage[ngerman]{babel}
\usepackage{xcolor-solarized}
\usetikzlibrary{calc}
\usetikzlibrary{positioning}
\usetikzlibrary{arrows}
\usetikzlibrary{calc}
\usetikzlibrary{fit}
\usetikzlibrary{spy}
\usetikzlibrary{backgrounds}
\usetikzlibrary{shapes.symbols}
\usepackage{dirtree}
\usepackage{minted}
\setminted{%
    %style=xcode,
    %style=trac,
    %style=paraiso-light,
    %style=lovelace,
    style=murphy,
    bgcolor=solarized-base3,
    linenos=true,
    autogobble,
}
\usepackage{booktabs}
%\setbeamercolor{alerted text}{fg=solarized-red}
%>>>
% ------------------------------------------------------------------- Macros <<<
% Arrow for use in text
\def\tikzrarrow{%
    \tikz[baseline=-0.67ex]\draw[double,very thick,-{stealth}](0,0)--(0.5,0);}
\newlength{\yOffs}
\newlength{\xOffs}
\newcommand*\code[1]{\texttt{#1}}
\newcommand\source[1]{%
    \tikz[remember picture,overlay]
        \node[align=left,anchor=south west,text=solarized-base1,font=\tiny]
        at (current page.south west)
        {Quelle: #1};
}
% Node for use in Table/Text
\newcommand\contrastnode[1]{%
    \tikz[
        baseline=(n0.base),
        rounded corners=2mm,
        inner sep=5pt,
        text=solarized-base2,
    ]
        \node[fill=solarized-base01] (n0) {#1};
}
%>>>
% -------------------------------------------------------------- Title Setup <<<
\title{\vspace*{4em}\Huge\LaTeX}
\subtitle{\hfill Trouble on the Horizon}
%\date{\today}
\date{26. April 2017}
\author{%
    Raphael Frey%
    \hfill%
    \scriptsize\texttt{%
        \href{mailto:rmfrey@runbox.com}%
        {rmfrey@runbox.com}}}

%\institute{IME}
\titlegraphic{\includegraphics[height=2em]{images/fhnwLogoDE-solarized-base02.eps}}
\subject{LaTeX -- Kurze Einf\"uhrung}
\keywords{LaTeX Introduction Primer FHNW Overview}
%http://web.fhnw.ch/cd/corporate-design/logos-fur-die-hochschulen
%>>>
%>>>
% **************************************************************************** %
\begin{document}
% **************************************************************************** %

\frame[plain]{\titlepage} % -------------------------------------- TITLE FRAME %


% ============================================================= BIBLIOGRAPHY <<<
\begin{frame} % ----------------------------------------------- BIBTEX INTRO <<<
    \frametitle{Bibliographie -- Wer, Wozu?}
    \centering
    \setlength\tabcolsep{1pt}
    \renewcommand\arraystretch{1.75}
    \begin{tabular}{p{20mm}rclp{34mm}}
        \toprule
        \scshape Layer    &
        \multicolumn{3}{l}{\scshape Komponente} &
        \scshape Zweck \\

        \midrule

        \LaTeX                  &
        \contrastnode{biblatex} &
        \contrastnode{natbib}   &
        \contrastnode{native}   &
        Bereitstellen von \mintinline{tex}|\cite|  and friends \\

        externer Arbeiter                &
        \contrastnode{\scshape Bib\TeX}  &
        \contrastnode{biber}             &
                                         &
        Verkn\"upfen von \code{.tex}- und \code{.bib}-Dateien \\

        Datenbank-Datei            &
        \contrastnode{\code{.bib}} &
        \contrastnode{andere}      &
                                   &
        Referenzdaten speichern    \\

        \bottomrule
    \end{tabular}
    \source{https://tex.stackexchange.com/a/299286/131649}
\end{frame}
%>>>
\begin{frame} % ------------------------------------------ BIBTEX FLOW CHART <<<
    \frametitle{Bib\TeX\ -- Was, Woher, Wohin?}
    \hspace*{-1em}%
    \begin{tikzpicture}[
        draw=solarized-base02,
        text=solarized-base02,
        rounded corners=2mm,
        every node/.append style={
            draw,
            thick,
            fill=solarized-base2,
            % technically not necessary with the global rounded corners option:
            rounded corners=2mm,
            inner sep=2mm,
            font=\fontsize{9}{10}\selectfont,
        },
        inputfile/.style={
            draw=solarized-orange,
            double,
        },
        signal/.append style={
            text=solarized-base2,
            fill=solarized-magenta,
            rounded corners=1mm,
        },
    ]
        \setlength{\xOffs}{5mm}
        \setlength{\yOffs}{6mm}

        \node[inputfile] (tex)
            {\code{.tex}};

        \node (latex1)
            [xshift=\xOffs,yshift=-\yOffs,rotate=-45,signal,shape = signal,signal from =west,signal to=east,below right of=tex]
            {latex};

        \node (aux1)
            [xshift=2*\xOffs,yshift=-\yOffs,right of=latex1]
            {\code{.aux}};

        \node (latex2)
            [xshift=2.5*\xOffs,signal,shape = signal,signal from =west,signal to=east,right of=aux1]
            {latex};

        \node (bibtex)
            [yshift=-2.33 * \yOffs,signal,shape = signal,signal from =west,signal to=east,below of=aux1]
            {bibtex};

        \node (aux2)
            [xshift=2.5*\xOffs,right of=latex2]
            {\code{.aux}};

        \node (latex3)
            [xshift=2*\xOffs,rotate=-45,signal,shape = signal,signal from =west,signal to=east,right of=aux2]
            {latex};

        \node (bbl)
            [yshift=-\yOffs,below of=latex2]
            {\code{.bbl}};

        \node[inputfile] (bib)
            [xshift=-9* \xOffs,yshift=-\yOffs,below of=latex2]
            {\code{.bib}};

        \node[inputfile] (bst)
            [yshift=-\yOffs,below of=bib]
            {\code{.bst}};

        \node (blg)
            [yshift=-\yOffs,below of=bbl]
            {\code{.blg}};

        \node (pdf)
            [xshift=8.5 * \xOffs,right of=blg]
            {\color{solarized-red}\code{.pdf,.dvi}};

        \draw[thick,-{stealth}]
            (tex.south) to[out=-90,in=135] (latex1.west);

        \draw[thick,-{stealth}]
            (tex.east) to[out=0,in=180] (latex2.west);

        \draw[thick,-{stealth}]
            (tex.east) to[out=0,in=135] (latex3.west);

        \draw[thick,-{stealth}]
            (latex1.east) to[out=-45,in=180] (aux1.west);

        \draw[thick,-{stealth}]
            (aux1.east) -- (latex2.west);

        \draw[thick,-{stealth}]
            (latex2.east) -- (aux2.west);

        \draw[thick,-{stealth}]
            (latex3.east) to[out=-45,in=90] (pdf.north);

        \draw[thick,-{stealth}]
            (bib.east) to[out=0,in=180] (bibtex.west);

        \draw[thick,-{stealth}]
            (bst.east) to[out=0,in=180] (bibtex.west);

        \draw[thick,-{stealth}]
            (bibtex.east) to[out=0,in=180] (bbl.west);

        \draw[thick,-{stealth}]
            (bbl.north) -- (latex2.south);

        \draw[thick,-{stealth}]
            (bibtex.east) to[out=0,in=180] (blg.west);

        \draw[thick,-{stealth}]
            (aux2.east) to[out=0,in=135] (latex3.west);

        \draw[thick,-{stealth}]
            (aux1.south) -- (bibtex);

    \end{tikzpicture}

    \source{Mittelbach et al.,\emph{The \LaTeX\ Companion}, 2nd Ed., p.688, Fig. 12.1}
\end{frame}
%>>>
\begin{frame}[fragile] % ------------------------- NATIVE BIBLIOGRAPHY: CODE <<<
    \frametitle{Bibliography -- Native}

    \inputminted{tex}{code/native.tex}
\end{frame}
%>>>
\begin{frame} % -------------------------------- NATIVE BIBLIOGRAPHY: RESULT <<<
    \frametitle{Bibliography -- Native}

    \fbox{\includegraphics[width=\textwidth,clip,viewport=45mm 220mm 170mm 257mm]{code/native.pdf}}
\end{frame}
%>>>
\begin{frame}[fragile] % ---------------------------- BIBTEX: REFERENCES.BIB <<<
    \frametitle{Bib\TeX}

    \code{references.bib}-File:

    \inputminted{tex}{code/references.bib}
\end{frame}
%>>>
\begin{frame}[fragile] % -------------------------------------- BIBTEX: CODE <<<
    \frametitle{Bib\TeX}

    \code{.tex}-File:

    \inputminted{tex}{code/bibtex.tex}
\end{frame}
%>>>
\begin{frame} % --------------------------------------------- BIBTEX: RESULT <<<
    \frametitle{Bib\TeX}

    \fbox{\includegraphics[width=\textwidth,clip,viewport=45mm 220mm 170mm 257mm]{code/bibtex.pdf}}
\end{frame}
%>>>
\begin{frame} % -------------------------------------------- BIBTEX: CAVEATS <<<
    \frametitle{Bib\TeX\ IEEEtran}

    \begin{itemize}
        \item
            \alert{Vorsicht:} Es    gibt     auch    eine    \textsf{IEEEtran}
            \LaTeX-Klasse! (\code{IEEEtran.cls})
        \item
            Wir   sind  aber   nur  am   \textsc{Bib}\TeX-Style  interessiert:
            \code{IEEEtran.\alert{bst}}
        \item
            Dokumentation:         \code{IEEEtran\_HOWTO.pdf}        respektive
            \code{IEEEtran\_bst\_HOWTO.pdf}
        \item
            \href{http://ctan.org/tex-archive/macros/latex/contrib/IEEEtran/bibtex}
                 {\nolinkurl{ctan.org/tex-archive/macros/latex/contrib/IEEEtran/bibtex}}
        \item
            \href{http://www.ieee.org/documents/IEEEtranBST.zip}
                 {\nolinkurl{ieee.org/documents/IEEEtranBST.zip}}
        \item
            \href{https://github.com/alpenwasser/TeX/tree/master/bibliography/official-IEEE-resources}
                 {\nolinkurl{https://github.com/alpenwasser/TeX/bibliography/official-IEEE-resources}}
    \end{itemize}
\end{frame}
%>>>
\begin{frame} % -------------------------------------------- BIBTEX: CAVEATS <<<
    \frametitle{Bib\TeX\ -- Die h\"assliche Seite}

    \begin{itemize}
        \item
            Referenzieren   und  Erstellen   der  Bibliographie   funktioniert
            $\pm$ automatisch, aber
        \item
            es    gibt    16    verschiedene   IEEE    citation    types    in
            \code{IEEEtran.bst}, und
        \item
            welcher Typ  unter welchen  Umst\"anden wie  auszuf\"ullen
            ist, ist eine Kunst f\"ur sich.
        \item
            Meinungsverschiedenheiten sind  vorprogrammiert, selbst  wenn alle
            beteiligten  Parteien   nach  bestem  Wissen  und   Gewissen  nach
            offiziellen Style Guides vorgehen.
    \end{itemize}

    \hspace*{-1em}\tikzrarrow\ Allenfalls mal im Voraus die Bibliographie absegnen lassen.
\end{frame}
%>>>
\begin{frame} % ---------------------- BIBTEX: ACCESS DATE, NUMBERED SECTION <<<
    \frametitle{Bib\TeX\ -- Abschliessende Feinheiten}
    \begin{description}
        \item[Platzierung:]
            Bibliographie     als     nummerierte     Section     nach     den
            \emph{Schlussfolgerungen},     \emph{Fazit}:       Aufgabe     des
            \code{.cls}-Files, momentan aber noch nicht implementiert.
        \item[Zugriffsdatum:]
            Wird      von     \code{IEEEtran.bst}      nicht     unterst\"utzt
            (implizite    Annahme:     alle    URLs    sind     g\"ultig    am
            Publikationsdatum). Workarounds:
            \begin{itemize}
                \item
                    Entweder globale Anmerkung diesbez\"uglich machen, oder
                \item
                    manuell das \code{.bbl}-File  editieren (Anh\"angen an den
                    Eintrag gem\"ass Style Guide m\"oglich).
            \end{itemize}
    \end{description}
\end{frame}
%>>>
%>>>


% ======================================================== MODULAR DOCUMENTS <<<
\begin{frame} % --------------------------------------------------- \input{} <<<
    \frametitle{Modulare Dokumente: \code{input}}
    \begin{center}
        \mintinline{tex}|\input{filename.tex}|
    \end{center}

    \begin{itemize}
        \item
            Importiert code aus \code{filename.tex} in das \"ubergeordnete
            Dokument und f\"uhrt ihn aus.
        \item
            Kann verschachtelt werden.
        \item
            Kann an ziemlich jedem Ort verwendet werden.
        \item
            Gut f\"ur Grafiken, Konfigurationsdateien geeignet.
    \end{itemize}
\end{frame}
%>>>
\begin{frame} % ------------------------------------------------- \include{} <<<
    \frametitle{Modulare Dokumente: \code{include}}
    \begin{center}
        \mintinline{tex}|\include{filename}|
    \end{center}

    \begin{itemize}
        \item
            Schiebt ein \code{\textbackslash clearpage} vor und nach
            \code{filename} ein. \tikzrarrow\ Neue Seiten.
        \item
            \"Offnet ein neues \code{.aux}-File
        \item
            Kann \alert{nicht} verschachtelt werden.
        \item
            Kann \alert{nicht} in der Preamble verwendet werden.
        \item
            Erlaubt selektives Ein-und Ausschliessen von Sub-Dokumenten.
        \item
            Geeignet  f\"ur  gr\"ossere  Bl\"ocke,  die  allenfalls  auch  mal
            unabh\"angig  vom Rest  des  Dokuments  kompiliert werden  sollten
            (Chapter, Section).
    \end{itemize}
\end{frame}
%>>>
\begin{frame}[fragile] % ------------------------------------ \includeonly{} <<<
    \frametitle{Modulare Dokumente: \code{includeonly}}
    \begin{minted}{tex}
        \includeonly{subdoc2,subdoc3}
        \begin{document}
        \section{Subdoc 1}
\label{subdoc1}

Reference to subdoc 4: \ref{subdoc4}, Page \pageref{subdoc4}

        \section{Subdoc 2}
\label{subdoc2}

Reference to subdoc 1: \ref{subdoc1}, Page \pageref{subdoc1}

        \section{Subdoc 3}
\label{subdoc3}

Reference to subdoc 1: \ref{subdoc1}, Page \pageref{subdoc1}

        \section{Subdoc 4}
\label{subdoc4}

Reference to subdoc 1: \ref{subdoc1}, Page \pageref{subdoc1}

    \end{minted}

    \begin{itemize}
        \item
            N\"utzlich, wenn  man nicht  immer das ganze  Dokument kompilieren
            will.
        \item
            Counter, Referenzen, Seitenzahlen  entsprechen dem Gesamtdokument,
            solange   dieses   einmal   kompiliert    worden   ist   (um   die
            \code{.aux}-Files zu f\"ullen).
    \end{itemize}
\end{frame}
%>>>
\begin{frame}[fragile] % ----------------------------------------- Practices <<<
    \frametitle{Modulare Dokumente: Some Practices}

    \begin{columns}
        \column{0.45\textwidth}
        Eine m\"ogliche Variante:
        {\small
        \dirtree{%
            .1 /.
            .2 Main.tex.
            .2 sections/.
            .3 intro.tex.
            .3 concept.tex.
            .3 implementation.tex.
            .3 conclusions.tex.
            .2 images/.
            .3 fancypic1.tex.
            .3 fancypic2.png.
        }}


        \column{0.55\textwidth}

        \begin{itemize}
            \item
                Root/Master-Dokument konfigurieren!
            \item
                Logisch strukturieren, nicht syntaktisch.
            \item
                Pfade in Subfiles  (z.B. \code{sections/intro.tex}) sind immer
                relativ zum Hauptdocument (hier \code{Main.tex}).
            \item
                Weitere Beispiele: Siehe \TeX\ Github-Repo.
        \end{itemize}
    \end{columns}
\end{frame}
%>>>
%>>>


% =============================================================== REFERENCES <<<
\begin{frame}[fragile] % ---------------------------------------- REFERENCES <<<
    \frametitle{Referenzieren}

    \begin{minted}[linenos=false]{tex}
        \label{marker}
        \ref{marker}
        \pageref{marker}
    \end{minted}

    \begin{itemize}
        \item
            \mintinline{tex}|\label|  muss  nach dem  zu  referenzierenden
            Objekt      kommen      (z.B.      \mintinline{tex}|\section|,
            \mintinline{tex}|\caption|).
        \item
            zwei Kompilierdurchl\"aufe erforderlich
        \item
            Ung\"ultige   Referenzen  werden   im  Output-Dokument   durch
            \code{??} ersetzt.
        \item
            Klickbare Links: Package \code{hyperref}.
    \end{itemize}

    \begin{center}
        z.B.: \mintinline{tex}|See Figure~\ref{fig} on page~\pageref{fig}|
    \end{center}
\end{frame}
%>>>
\begin{frame}[fragile] % ------------------------ REFERENCES: BEST PRACTICES <<<
    \frametitle{Referenzieren -- Best Practices}

    \centering

    \code{marker} folgendermassen definieren:

    \vspace*{1em}

    \begin{tabular}{ll|p{30mm}l}
        \toprule
        \scshape Objekt & \scshape Label Prefix & \scshape Objekt      & \scshape Label Prefix \\
        \midrule
        chapter         & \code{ch:}            & section              & \code{sec:}           \\
        subsection      & \code{subsec:}        & figure               & \code{fig:}           \\
        table           & \code{tab:}           & equation             & \code{eq:}            \\
        code listing    & \code{lst:}           & enumerated list item & \code{itm:}           \\
        algorithm       & \code{alg:}           & appendix subsection  & \code{app:}           \\
        \bottomrule
    \end{tabular}

    \begin{center}
        z.B.: \mintinline{tex}|\label{fig:la-tete-de-robespierre}|
    \end{center}
\end{frame}
%>>>
%>>>


% =================================================================== FLOATS <<<
\begin{frame} % ------------------------------------------------FLOATS INTRO <<<
    \frametitle{Was sind Floats?}

    \begin{itemize}
        \item
            Normaler Text wird  von \TeX\ gesammelt und  an geeigneten Stellen
            in Zeilen zerlegt.
        \item
            Macht z.B. f\"ur Bilder und Tabellen nicht viel Sinn.
        \item
            Stattdessen   sollten  diese   so  platziert   werden,  dass   ein
            m\"oglichst optimales Layout resultiert.
    \end{itemize}

    \tikzrarrow\ L\"osung: floats (``Gleitumgebungen'')
\end{frame}
%>>>
\begin{frame}[fragile] % ------------------------------ FLOATS: FIGURE/TABLE <<<
    \frametitle{Standard-Floats}

    \begin{columns}
        \column{0.4\textwidth}
        \begin{itemize}
            \item
                \code{figure}
            \item
                \code{figure*}
            \item
                \code{table}
            \item
                \code{table*}
        \end{itemize}

        \column{0.6\textwidth}
        \begin{minted}{tex}
            \begin{table}
                \centering
                \begin{tabular}{ll}
                    A & a \\
                    B & b \\
                \end{tabular}
                \caption{A table}
                \label{tab:a-table}
            \end{table}
        \end{minted}
    \end{columns}
\end{frame}
%>>>
\begin{frame}[fragile] % ---------------------------------- FLOATS: CAPTIONS <<<
    \frametitle{Captions}

    \begin{minted}{tex}
        \caption[Short Caption]{%
            Long Caption Text with explanations,
            elaborations, clarifications and
            contemplations.}
        \label{tab:a-table}
    \end{minted}

    \begin{itemize}
        \item
            Optionales Argument: \mintinline{tex}|[Short Caption]|: F\"ur
            \emph{List of Figures} etc.
        \item
            \mintinline{tex}|\label|:            \alert{Nach}            dem
            \mintinline{tex}|\caption|-Command
    \end{itemize}
\end{frame}
%>>>
\begin{frame}[fragile] % -------------------------------- NON-FLOAT CAPTIONS <<<
    \frametitle{Captions ohne Floats}

    \begin{minted}{tex}
        \usepackage{capt-of}
        ...
        \captionof{table}{%
            This is a table with things in it.}
        \label{tab:table-with-things}
    \end{minted}

    Vorsicht  mit   Reihenfolge  beim  Mischen  von   Floats  und  Non-Floats.
    \tikzrarrow\   Allenfalls   \mintinline{tex}|\clearpage|   oder   sonstige
    manuelle Eingriffe erforderlich.
\end{frame}
%>>>
\begin{frame}[fragile] % -------------------------- FLOATS: PLACEMENT PARAMS <<<
    \frametitle{Placement Options}

    Teilen \LaTeX\ mit, wo das Platzieren des entsprechenden Floats erlaubt ist.

    \begin{columns}
        \column{0.33\textwidth}
        \begin{itemize}
            \item[\code{h}]: \emph{Here}
            \item[\code{t}]: \emph{Top}
            \item[\code{b}]: \emph{Bottom}
            \item[\code{p}]: \emph{Page of floats}
            \item[\code{!}]: \emph{Try Harder}
        \end{itemize}
        \column{0.67\textwidth}

        \begin{itemize}
            \item
                default: \mintinline{tex}|\begin{figure}[tbp]|
            \item
                alternativ, z.B.: \mintinline{tex}|\begin{figure}[!htp]|
            \item
                Aufpassen,  dass man  \LaTeX\ gen\"ugend  Flexibilit\"at gibt,
                sonst kann  es passieren,  dass floats  bis Ende  des Kapitels
                gebuffert werden.
            \item
                Erzwingen der Ausgabe: \mintinline{tex}|\clearpage|,
                \mintinline{tex}|\cleardoublepage|
        \end{itemize}

    \end{columns}
\end{frame}
%>>>
\begin{frame} % ------------------------------------------- FLOATS: PROBLEMS <<<
    \frametitle{Floats -- Probleme}

    \begin{itemize}
        \item
            viele Floats, wenig Text
        \item
            Alle        floats       springen        ans       Ende        des
            Kapitels/Dokuments/\ldots: Normalerweise  verursacht  durch  einen
            Mangel an Placement Options f\"ur ein oder mehrere Floats.
        \item
            Andere     unerw\"unschte     Ergebnisse: Manuelle     Workarounds
            erforderlich (siehe \TeX\ Github-Repo, Float-Tutorial, Abschn. 5).
    \end{itemize}
\end{frame}
%>>>
%>>>


% ================================================================= LISTINGS <<<
\begin{frame} % ----------------------------------------- LISTINGS: OVERVIEW <<<
    \frametitle{Listings -- \"Ubersicht}

    \begin{itemize}
        \item
            viele verschiedene  Packages  mit unterschiedlichen  St\"arken und
            Schw\"achen
        \item
            Stock     \code{verbatim}     environment,     Packages,     u.a.:
            \code{verbatimbox},        \code{fancyvrb},        \code{listing},
            \code{listings}, \code{matlab-prettifier}, \code{minted}, \ldots
        \item
            Siehe
            \href{https://github.com/alpenwasser/TeX/tree/master/listings}
                 {\nolinkurl{https://github.com/alpenwasser/TeX/tree/master/listings}}
            f\"ur eine kompakte Einf\"uhrung mit einigen Beispielen.
        \item
            (Diese Pr\"asentation verwendet \code{minted}.)
    \end{itemize}
\end{frame}
%>>>
\begin{frame} % ----------------------------------------- LISTINGS: OVERVIEW <<<
    \frametitle{Listings -- Hinweise}

    \begin{itemize}
        \item
            Es   gibt  zwei   Grundprinzipien: Inline  Content   und  separate
            Environments.
        \item
            Bei       Environments: Darauf        achten,       dass       das
            \mintinline{tex}|\end{environment}|  auf   einer  separaten  Zeile
            steht, ohne Kommentar oder sonstiges Beigem\"use.
        \item
            Gewisse  packages  unterst\"utzen Escape-Mechanismen,  um  manuell
            in   den  Formatierungsprozess   eingreifen   zu  k\"onnen   (z.B.
            \code{escapeinside}  bei  \code{minted},  \code{commandchars}  bei
            \code{fancyvrb}).
        \item
            F\"ur ``Fancy Features'': Manuals konsultieren.
    \end{itemize}
\end{frame}
%>>>
%>>>


% =================================================================== FLOATS <<<
\begin{frame} % ---------------------------------------- MINIPAGES: OVERVIEW <<<
    \frametitle{minipages}

    Praktisch, um z.B. \ldots
    \begin{itemize}
        \item
            \ldots\ Dinge nebeneinander zu platzieren,
        \item
            als Alternative zu Floats,
        \item
            wenn man Dinge beisammen behalten muss, die sich trennen k\"onnten.
        \item
            Siehe
            \href{https://github.com/alpenwasser/TeX/tree/master/minipages}
                 {\nolinkurl{https://github.com/alpenwasser/TeX/tree/master/minipages}}
    \end{itemize}
\end{frame}
%>>>
\begin{frame}[fragile] % -------------------------------- MINIPAGES: EXAMPLE <<<
    \frametitle{minipages: Beispiel}

    Inhalte nebeneinander platzieren:

    \begin{minted}{tex}
        \begin{minipage}{0.45\textwidth}
            Robespierre
        \end{minipage}
        \hfill
        \begin{minipage}{0.45\textwidth}
            Robespierre's head
        \end{minipage}
    \end{minted}

    Ergebnis:

    \fbox{\begin{minipage}{0.45\textwidth}
        Robespierre
    \end{minipage}}
    \hfill
    \fbox{\begin{minipage}{0.45\textwidth}
        Robespierre's head
    \end{minipage}}
\end{frame}
%>>>
%>>>


\begin{frame} % -------------------------------------- NICE-TO-HAVE PACKAGES <<<
    \frametitle{Lohnenswerte Packages}
    \begin{itemize}
        \item hyperref
        \item microtype
        \item babel/polyglossia
        \item inputenc
        \item amsmath
        \item fontenc, fontspec, luatextra: https://tex.stackexchange.com/a/44701/131649
    \end{itemize}
\end{frame}
%>>>


\begin{frame} % ------------------------------------------------------ LINKS <<<
    \frametitle{Links}
    \begin{itemize}
        \item
            \alert{\href{https://github.com/alpenwasser/TeX/}
                        {https://github.com/alpenwasser/TeX/}}
        \item
            Modulare Dokumente:
            \begin{itemize}
                \item
                    \href{https://en.wikibooks.org/wiki/LaTeX/Modular_Documents}
                         {\nolinkurl{en.wikibooks.org/wiki/LaTeX/Modular_Documents}}
                 \item
                     \href{https://tex.stackexchange.com/questions/246/}
                          {\nolinkurl{tex.stackexchange.com/questions/246/}}
            \end{itemize}
        \item
            Referenzieren:
            \href{https://en.wikibooks.org/wiki/LaTeX/Labels_and_Cross-referencing}
                 {\nolinkurl{https://en.wikibooks.org/wiki/LaTeX/Labels_and_Cross-referencing}}
    \end{itemize}
\end{frame}
%>>>

%\begin{frame} % -------------------------------------------------------- FRAME %
%    \frametitle{Daten Plotten}
%    \begin{itemize}
%        \item matlab2tikz
%        \item pgfplots
%    \end{itemize}
%\end{frame}

%\begin{frame} % -------------------------------------------------------- FRAME %
%    \frametitle{Einfache Macros}
%\end{frame}

%\begin{frame} % -------------------------------------------------------- FRAME %
%    \frametitle{Farben}
%\end{frame}
%
%\begin{frame} % -------------------------------------------------------- FRAME %
%    \frametitle{Externe PDF-Dokumente}
%\end{frame}
%
%\begin{frame} % -------------------------------------------------------- FRAME %
%    \frametitle{Elektrische Schaltungen zeichnen}
%\end{frame}
%
%
%
%\begin{frame} % -------------------------------------------------------- FRAME %
%    \frametitle{Support}
%
%    \begin{itemize}
%        \item
%            e-Mail:
%            \texttt{\href{mailto:raphael.frey@students.fhnw.ch}
%            {raphael.frey@students.fhnw.ch}}
%        \item
%            Raum 4.223
%        \item
%            \alert{\href{https://github.com/alpenwasser/TeX/}
%                        {https://github.com/alpenwasser/TeX/}}
%        \item
%            Weiteres Programm: 26. April
%            \begin{itemize}
%                \item
%                    modulare Dokumentstruktur
%                \item
%                    Bibliographie
%                \item
%                    Floats (Tabellen, Bilder, \ldots )
%                \item
%                    PGF/Ti\emph{k}Z
%                %\item
%                %    minipages
%                %\item
%                %    ISO-31
%                \item
%                    Listings, minted, \ldots
%                \item
%                    etc. etc. etc.
%            \end{itemize}
%    \end{itemize}
%\end{frame}

% ---------------------------------------------------------------------------- %
\section<handout:0| trans:0>*{Fragen?} % ------------------------------- FRAME %
% ---------------------------------------------------------------------------- %
\end{document}
% vim: foldenable foldcolumn=4 foldmethod=marker foldmarker=<<<,>>>
