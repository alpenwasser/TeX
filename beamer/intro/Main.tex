% ******************************************************** %
% DOCUMENT INFORMATION                                     %
% ******************************************************** %
%                                                          %
% Purpose of Document                                      %
% -------------------                                      %
% Introductory Presentation on LaTeX                       %
%                                                          %
% Institution                                              %
% -----------                                              %
% University of Applied Sciences and Arts Northwestern     %
% Switzerland, School of Engineering                       %
%                                                          %
% Degree Program                                           %
% --------------                                           %
% Electrical Engineering and Information Technology, BSc.  %
%                                                          %
% Author & Copyright                                       %
% ------------------                                       %
% Raphael Frey, raphael.frey@students.fhnw.ch              %
%               rmfrey@runbox.com                          %
%                                                          %
% Date: 2017-MAR-15                                        %
% ******************************************************** %

\documentclass{beamer}                % Presentation Version
%\documentclass[handout]{beamer}      % Handout Version

% Beamer Setup --------------------------------------------------------------- %
%\addtobeamertemplate{background canvas}{\transdissolve[duration=1]}{}
\usetheme[titleformat=smallcaps,numbering=none]{metropolis}

% Packages ------------------------------------------------------------------- %
\usepackage[ngerman]{babel}
\usepackage{xcolor-solarized}
\usetikzlibrary{calc}
\usetikzlibrary{positioning}
\usetikzlibrary{arrows}
\usetikzlibrary{calc}
\usetikzlibrary{fit}
\usetikzlibrary{backgrounds}
\usetikzlibrary{shapes.symbols}
\usepackage{hyperref}
\usepackage{dirtree}
\usepackage{minted}
\setminted{%
    %style=xcode,
    %style=trac,
    %style=paraiso-light,
    %style=lovelace,
    style=murphy,
    bgcolor=solarized-base3,
    linenos=true,
}
%\setbeamercolor{alerted text}{fg=solarized-red}


% Macros --------------------------------------------------------------------- %
% Arrow for use in text
\def\tikzrarrow{%
    \tikz[baseline=-0.67ex]\draw[double,very thick,-{stealth}](0,0)--(0.5,0);}
\newlength{\yOffs}
\newlength{\xOffs}

% Title Setup ---------------------------------------------------------------- %
\title{\vspace*{4em}\Huge\LaTeX}
\date{\today}
\author{%
    Raphael Frey%
    \hfill%
    \footnotesize\texttt{%
        \href{mailto:raphael.frey@students.fhnw.ch}%
        {raphael.frey@students.fhnw.ch}}}

\institute{\vspace*{7em}\hfill\includegraphics[height=2em]{images/fhnw.eps}}

% **************************************************************************** %
\begin{document}                                                              
% **************************************************************************** %

\maketitle % ----------------------------------------------------------- FRAME %


\begin{frame}<handout:0> % --------------------------------------------- FRAME %
    \frametitle{Programm}
    \tableofcontents
\end{frame}


% ---------------------------------------------------------------------------- %
\section<handout:0>{Was und Warum?} % ---------------------------------- FRAME %
% ---------------------------------------------------------------------------- %


\begin{frame} % -------------------------------------------------------- FRAME %
    \frametitle{Was ist \LaTeX?}

    \begin{tikzpicture}[
        draw=solarized-base02,
        rounded corners=2mm,
        text=solarized-base02,
        every node/.append style={
            draw,
            thick,
            fill=solarized-base2,
            % technically not necessary with the global rounded corners option:
            rounded corners=2mm, 
            inner sep=2mm,
            font=\fontsize{10}{11}\selectfont,
        },
    ]
        \setlength{\yOffs}{14mm}
        \setlength{\xOffs}{4mm}

        % TeX Node
        \node (tex) [align=left] at (0,0) {%
            \Large\textbf{\alert{\TeX}}\\[2mm]
            Donald Knuth, 1978\\
        };

        % LaTeX Node
        \pause
        \node (latex)
            [below right=\yOffs and \xOffs of tex,align=left,anchor=west]
            {\Large\textbf{\alert{\LaTeX}}\\[2mm] Leslie Lamport, 1985};
        \draw
            [thick,-{stealth}] 
            (tex) -| (latex.north);

        % ConTeXt Node
        \pause
        \node (context) 
            [below left=\yOffs and \xOffs of tex,align=left,anchor=east]
            {\Large\textbf{\alert{ConTeXt}}\\[2mm] Hans Hagen, 1991};
        \draw
            [thick,-{stealth}] 
            (tex) -| (context.north);

        % Others...
        \pause
        \node (others)
            [below=\yOffs of tex,align=left,anchor=center] 
            {weitere\ldots};
        \draw
            [thick,-{stealth}] 
            (tex) -- (others.north);

        % Auxiliary Coordinate
        \pause
        \coordinate [below=40mm of tex] (southAux);
        % NOTE:
        % http://tex.stackexchange.com/questions/57227
        \node (contamination) at (southAux)
            [text=solarized-base0,yshift=8mm,anchor=south,draw=none,fill=none]
            {gegens. Kontaminierung};

        \draw
            [solarized-base0,dashed,thick,{stealth}-{stealth}] 
            (others.west) -- (context.east);
        \draw
            [solarized-base0,dashed,thick,{stealth}-{stealth}] 
            (others.east) -- (latex.west);
        \draw
            [solarized-base0,dashed,thick,{stealth}-] 
            (context.south) |- (southAux);
        \draw
            [solarized-base0,dashed,thick,{stealth}-]
            (latex.south) |- (southAux);
    \end{tikzpicture}
\end{frame}

\begin{frame}<handout:0> % ----------------------------------------------FRAME %
    \frametitle{Warum nicht?}


    \begin{itemize}
        \item
            Word \& Co. funktionieren heutzutage ja egtl. nicht schlecht \ldots
        \item
            \ldots und auch \TeX~hat seine T\"ucken:
            \begin{itemize}
                \item
                    Floats
                \item
                    Inkompatibilit\"aten zwischen Packages
                \item
                    Schriftarten
                \item
                    Input Encoding, Output Encoding
                \item
                    \"Okosystem ist gross und etwas un\"ubersichtlich
                \item
                    umst\"andlicher    Workflow     (mehrere    Compiler    \&
                    Compiler-Durchl\"aufe, verschiedene Optionen)
                \item
                    und so weiter, und so fort
            \end{itemize}
    \end{itemize}
    \pause
    \centering\alert{\tikzrarrow Ein gewisses Mass an Masochismus ist erforderlich.}
\end{frame}

\begin{frame}<handout:0>
    \frametitle{Warum doch?}
    \begin{itemize}
        \item
            Trennung von Inhalt und Form 
            %(m\"oglich, aber h\"aufig nicht sauber befolgt)
        \item
            sehr m\"achtig: Packages f\"ur jeden Geschmack
        \item
            Referenzen: Fussnoten,    Randnotizen,   Abbildungen,    Tabellen,
            Gleichungen, \ldots
            % Word kann das auch. Aber in meiner Erfahrung mehr quirky behavior.
        \item
            robusteres Verhalten (wenn auch nicht immer so wie gewollt)
        \item
            nicht propriet\"ar
        \item
            Basiert auf Puretext-Dateien:
            \begin{itemize}
                \item
                    Version Control: Git, SVN \& Co.
                \item
                    Modularit\"at
                \item
                    zukunftssicher (Archivierbarkeit)
            \end{itemize}
        \item
            die feineren Aspekte von Typographie
    \end{itemize}
\end{frame}

% ---------------------------------------------------------------------------- %
\section<handout:0>{Wie? -- Einrichtung \& Workflow} % ----------------- FRAME %
% ---------------------------------------------------------------------------- %

\begin{frame} % -------------------------------------------------------- FRAME %
    \frametitle{\TeX~Distributionen}

    Wie bekomme ich \TeX~ auf mein System?\\[2mm]

    \begin{center}
        \begin{tabular}{lll}
            TeX Live                                                          & 
            Win                                                               & 
            \href{https://www.tug.org/texlive/}{https://www.tug.org/texlive/} \\

                                                                              &
            *nix                                                              &
            Package Manager                                                   \\

            Miktex                                                            &
            Win                                                               & 
            \href{https://miktex.org}{https://miktex.org}                     \\

            MacTeX                                                            & 
            OS X                                                              & 
            \href{https://www.tug.org/mactex/}{https://www.tug.org/mactex/}   \\

            \\

            \textcolor{solarized-base0}{andere}                               & 
            \multicolumn{2}{l}{%
                \textcolor{solarized-base0}{(esoterisch und/oder obsolet)}}   \\
        \end{tabular}

    \pause
    \vspace{1em}
    \alert{\tikzrarrow siehe Anleitungen}
    \end{center}
\end{frame}

\begin{frame} % -------------------------------------------------------- FRAME %
    \frametitle{Workflow}
    \centering
    \begin{tikzpicture}[
        draw=solarized-base02,
        rounded corners=2mm,
        text=solarized-base02,
        every node/.append style={
            draw,
            thick,
            fill=solarized-base2,
            inner sep=2mm,
            font=\fontsize{10}{11}\selectfont,
            align=left,
        },
        signal/.append style={
            text=solarized-base2,
            fill=solarized-magenta,
            rounded corners=1mm,
        },
    ]

        \node (source) 
            [align=left] 
            at (0,0)
            {Quellcode};

        \node (pdflatex) 
            [signal,shape=signal,signal from=west,signal to=east,right=20mm of source]
            {pdflatex};

        \node (pdf) 
            [right=20mm of pdflatex]
            {pdf};

        \node (latex) 
            [signal,shape = signal,signal from=west,signal to=east,
            below=of source,yshift=-10mm,rotate=-90,anchor=center] 
            {latex};

        \node (dvi) 
            [below=40 mm of source]
            {DVI};

        \node (dvips)
            [signal,shape=signal,signal from=west,
            signal to=east,below=40mm of pdflatex]
            {dvips};

        \node (postscript) 
            [below=40mm of pdf,]
            {PostScript};

        \node (ps2pdf)
            [signal,shape = signal,signal from=west,signal to=east,
            below=of pdf,yshift=-10mm,rotate=90,anchor=center]
            {ps2pdf};

        \begin{scope}[thick,arrows={-{stealth[solarized-base02]}}]
            \draw (source) -- (pdflatex);
            \draw (source) -- (latex);
            \draw ($(pdflatex.east) + (-0.5mm,0)$) -- (pdf);
            \draw ($(latex.east) + (0,0.5mm)$) -- (dvi);
            \draw (dvi) -- (dvips);
            \draw ($(dvips.east) + (-0.5mm,0)$) -- (postscript);
            \draw (postscript) -- (ps2pdf);
            \draw ($(ps2pdf.east) + (0,-0.5mm)$) -- (pdf);
        \end{scope}

        \pause
        \begin{scope}[
            draw=solarized-base02!20!white,
            very thick,
            arrows={-{stealth[solarized-base02!20!white]}},
            rounded corners=2mm,
            text=solarized-base02!20!white,
            every node/.append style={
                draw,
                thick,
                fill=solarized-base2!50!white,
                inner sep=2mm,
                font=\fontsize{10}{11}\selectfont,
                align=left,
            },
            signal/.append style={
                text=solarized-base2!60!solarized-magenta,
                fill=solarized-magenta!10!white,
                rounded corners=1mm,
            },
        ]

            \node (latex) 
                [signal,shape = signal,signal from=west,signal to=east,
                below=of source,yshift=-10mm,rotate=-90,anchor=center] 
                {latex};

            \node (dvi) 
                [below=40 mm of source]
                {DVI};

            \node (dvips) 
                [signal,shape = signal,signal from=west
                ,signal to=east,below=40mm of pdflatex]
                {dvips};

            \node (postscript) 
                [below=40mm of pdf,]
                {PostScript};

            \node (ps2pdf) 
                [signal,shape=signal,signal from=west,signal to=east,
                below=of pdf,yshift=-10mm,rotate=90,anchor=center]
                {ps2pdf};

        \end{scope}

        \begin{scope}[
                very thick,
                draw=solarized-base02!20!white,
                arrows={-{stealth[solarized-base02!20!white]}},
        ]
            \draw (source) -- (latex);
            \draw ($(latex.east) + (0,0.5mm)$) -- (dvi);
            \draw (dvi) -- (dvips);
            \draw ($(dvips.east) + (-0.5mm,0)$) -- (postscript);
            \draw (postscript) -- (ps2pdf);
            \draw ($(ps2pdf.east) + (0,-0.5mm)$) -- (pdf);
        \end{scope}

        % Additional Compilers, for the fitting node
        \node (xelatex) 
            [signal,shape = signal,signal from=west,
            signal to=east,below=1mm of pdflatex]
            {xelatex};

        \node (lualatex) 
            [signal,shape=signal,signal from=west,
            signal to=east,above=1mm of pdflatex] 
            {lualatex};

        \node (bibtex) 
            [signal,shape = signal,signal from=east,
            signal to=west,below=10mm of xelatex]
            {bibtex};

        % Using   the  'on   background  layer'  breaks  the
        % pre-pause slide, so we don't.
        \node (compilers) 
            [draw,fill=solarized-base2,fit=(lualatex) (pdflatex) (xelatex),
            inner sep=1em,xshift=-0.5em,align=right]
            {};

        % And  we  draw  this  stuff  twice,  so  that  it's
        % actually visible.
        \node (pdflatexTop) 
            [signal,shape = signal,signal from=west,
            signal to=east,right=20mm of source]
            {pdflatex};

        \node (xelatexTop) 
            [signal,shape = signal,signal from=west,
            signal to=east,below=1mm of pdflatex]
            {xelatex};

        \node (lualatexTop) 
            [signal,shape = signal,signal from=west,
            signal to=east,above=1mm of pdflatex]
            {lualatex};

        \node (bibtexTop) 
            [signal,shape = signal,signal from=east,
            signal to=west,below=10mm of xelatex]
            {bibtex};

        \draw ($(compilers.south east) + (-0.7mm,0.7mm)$)
            edge[thick,arrows={-{stealth[solarized-base2]}},out=-45,in=0] 
            ($(bibtex.east) + (0.3mm,0)$);
        \draw ($(bibtex.west) + (0.5mm,0)$)
            edge[thick,arrows={-{stealth[solarized-base2]}},out=180,in=225] 
            ($(compilers.south west) + (0.7mm,0.7mm)$);

        \begin{scope}[thick,arrows={-{stealth[solarized-base02]}}]
            \draw (source) -- (compilers.west);
            \draw (compilers.east) -- (pdf);
        \end{scope}
    \end{tikzpicture}
\end{frame}


\begin{frame} % -------------------------------------------------------- FRAME %
    \frametitle{Editoren}
    \begin{itemize}
        \item
            TeX Live \& MikTeX: TeXworks
        \item
            macOS: z.B. TeXShop
        \item
            Texmaker, TeXstudio, \ldots
        \item
            WYSIWYM: LyX
            %(What you see is what you mean)

        \item
        Vergleichstabelle: 
            \href
                {https://en.wikipedia.org/wiki/Comparison\_of\_TeX\_editors}
                {https://en.wikipedia.org/wiki/Comparison\_of\_TeX\_editors}

        \item
        Liste \& Diskussion:
            \href
                {http://tex.stackexchange.com/questions/339/}
                {http://tex.stackexchange.com/questions/339/}
    \end{itemize}

    \pause
    \centering\alert
        {abh\"angig von pers\"onlichen Vorlieben \tikzrarrow Ausprobieren}
\end{frame}

% ---------------------------------------------------------------------------- %
\section<handout:0>{Beispiel: Ein einfaches \LaTeX-Dokument} % --------- FRAME %
% ---------------------------------------------------------------------------- %


\begin{frame}[fragile] % ----------------------------------------------- FRAME %
    \frametitle{Einfachstes Grundger\"ust}

    \begin{columns}
        \column{0.5\textwidth}

        \inputminted{tex}{code/simplest.tex}

        \column{0.5\textwidth}
        \begin{tikzpicture}[draw=solarized-base02]
            \draw
                [-{stealth},very thick] 
                (0,0) -- (0.5,0);

            \node
                [anchor=west,draw] 
                at (1,0) 
                {\includegraphics[height=5.5cm]{code/simplest.pdf}};
        \end{tikzpicture}
    \end{columns}
\end{frame}

\begin{frame}[fragile] % ----------------------------------------------- FRAME %
    \frametitle{Optionale Parameter, Packages, Environments}

    \begin{columns}

        \column{0.5\textwidth}
        \begin{overprint} % ---------------------------------- OVERPRINT %
            \onslide<1|handout:0>
            \inputminted{tex}{code/packages.tex}

            \onslide<2|handout:0>
            % ----------------------------------------------- MINTED %
            \begin{minted}[escapeinside=||]{tex}
\documentclass
    |\alert{[a4paper,12pt]}|
    {article}
\usepackage{lipsum}
\begin{document}
\lipsum[1-2]
\begin{equation}
    a = 5
\end{equation}
\lipsum[3]
\end{document}
            \end{minted} 
            % ------------------------------------------------------ %

            \onslide<3|handout:0>
            % ----------------------------------------------- MINTED %
            \begin{minted}[escapeinside=||]{tex}
\documentclass
    [a4paper,12pt]
    {article}
|\alert{\textbf{\textbackslash\!\,usepackage\{lipsum\}}}|
\begin{document}
\lipsum[1-2]
\begin{equation}
    a = 5
\end{equation}
\lipsum[3]
\end{document}
            \end{minted}
            % ------------------------------------------------------ %

            \onslide<4|handout:0>
            % ----------------------------------------------- MINTED %
            \begin{minted}[escapeinside=||]{tex}
\documentclass
    [a4paper,12pt]
    {article}
\usepackage{lipsum}
\begin{document}
\lipsum[1-2]
|\alert{\textbf{\textbackslash\!\,begin\{equation\}}}|
    a = 5
|\alert{\textbf{\textbackslash\!\,end\{equation\}}}|
\lipsum[3]
\end{document}
            \end{minted}
            % ------------------------------------------------------ %

            \onslide<5->
            % ----------------------------------------------- MINTED %
            \begin{minted}[escapeinside=||]{tex}
\documentclass
    |\alert{[a4paper,12pt]}|
    {article}
|\alert{\textbf{\textbackslash\!\,usepackage\{lipsum\}}}|
\begin{document}
\lipsum[1-2]
|\alert{\textbf{\textbackslash\!\,begin\{equation\}}}|
    a = 5
|\alert{\textbf{\textbackslash\!\,end\{equation\}}}|
\lipsum[3]
\end{document}
            \end{minted}
            % ------------------------------------------------------ %
        \end{overprint} % ---------------------------------------------- %

        \column{0.5\textwidth}
        \begin{overprint} % ---------------------------------- OVERPRINT %
            \onslide<5->
            \begin{tikzpicture}[draw=solarized-base02]
                \draw
                    [-{stealth},very thick]
                    (0,0) -- (0.5,0);
                \node
                    [anchor=west,draw]
                    at (1,0)
                    {\includegraphics[height=5.5cm]{code/packages.pdf}};
            \end{tikzpicture}
        \end{overprint} % ---------------------------------------------- %
    \end{columns}

    \pause
    \begin{overprint}
        \onslide<6>
        \begin{center}
            \vspace{-1em}
            \alert{\tikzrarrow Stackexchange, Manuals}
        \end{center}
    \end{overprint}
\end{frame}

\begin{frame}[fragile] % ----------------------------------------------- FRAME %
    \frametitle{Dokumentstruktur}

    \inputminted{tex}{code/structure.tex}
\end{frame}

\begin{frame} % -------------------------------------------------------- FRAME %
    \frametitle{Version Control}

    Bin\"are  Dateien  erzeugen  gerne   Merge-Konflikte,  wenn  sie  h\"aufig
    abge\"andert werden. ``Popul\"are'' Kandidaten sind:

    \begin{itemize}
        \item
            \texttt{*.synctex.gz}-Datei: Dient  zur  Synchronisation  zwischen
            Quellcode und der zugeh\"origen \texttt{*.pdf}-Datei.
        \item
            Alle \texttt{*.pdf}-Dateien, die aus \LaTeX~kompiliert werden.
    \end{itemize}

    \emph{Empfehlung}: Diese Dateien zu \texttt{.gitignore} hinzuf\"ugen. 

    Nicht  zu empfehlen  f\"ur  Bilder  und andere  bin\"are  Dateien, die  im
    \LaTeX-Dokument eingebunden sind.
\end{frame}

% ---------------------------------------------------------------------------- %
\section<handout:0>{N\"achste Schritte} % ------------------------------ FRAME %
% ---------------------------------------------------------------------------- %

\begin{frame} % -------------------------------------------------------- FRAME %
    \frametitle{Disposition}

    FHNW Report Klasse: 
    \alert{%
        \href{http://public.ime.fhnw.ch/LaTeX/}
        {http://public.ime.fhnw.ch/LaTeX/}}\\

    \dirtree{%
        .1 fhnwreport\_with\_example\_v22.zip.
        .2 beispiel\_fachbericht.pdf.
        .2 beispiel\_fachbericht.tex.
        .2 example.bib.
        .2 fhnw\_ht\_logo\_de.pdf.
        .2 fhnw\_ht\_logo\_en.pdf.
        .2 fhnwlogo.pdf.
        .2 fhnwreport.cls.
        .2 IEEEabrv.bib.
        .2 IEEEfull.bib.
        .2 IEEEtran.bst.
    }

\end{frame}

\begin{frame} % -------------------------------------------------------- FRAME %
    \frametitle{Weitere Ressourcen}

    {\footnotesize\selectfont
    \begin{tabular}{ll}
        \href{http://tex.stackexchange.com/}{http://tex.stackexchange.com/} &
        (\"ublicherweise via Suchmaschine)                                  \\[2mm]

        \href{http://texample.net}{http://texample.net}                     &
        (TikZ und PGF)                                                      \\[2mm]

        \href{http://pgfplots.net}{http://pgfplots.net}                     &
        (PGF Plots)                                                         \\[2mm]

        \href{http://ctan.org/}{http://ctan.org}                            &
        (Package Dokumentationen)                                           \\[2mm]

        \href{https://www.overleaf.com}{https://www.overleaf.com}           &
        (Collaborative Editing, Free \& Paid Plans)                         \\[2mm]
    \end{tabular}}
\end{frame}

\begin{frame} % -------------------------------------------------------- FRAME %
    \frametitle{Support}

    \begin{itemize}
        \item
            e-Mail:
            \texttt{\href{mailto:raphael.frey@students.fhnw.ch}
            {raphael.frey@students.fhnw.ch}}
        \item
            Raum 4.2-irgendwas
        \item
            \href{https://github.com/alpenwasser/latex-sandbox/}
                 {https://github.com/alpenwasser/latex-sandbox/}
        \item
            Weiteres Programm: 26. April: Spezifische Probleme
            \begin{itemize}
                \item
                    modulare Dokumentstruktur
                \item
                    Bibliographie
                \item
                    Floats (Tabellen, Bilder, \ldots )
                \item
                    PGF/TikZ
                \item
                    minipages
                \item
                    ISO-31
                \item
                    Listings, minted, \ldots
                \item
                    etc. etc. etc.
            \end{itemize}
    \end{itemize}
\end{frame}

% ---------------------------------------------------------------------------- %
\section<handout:0>*{Fragen?} % ---------------------------------------- FRAME %
% ---------------------------------------------------------------------------- %
\end{document}
